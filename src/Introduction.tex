% --- Dokumentenklasse ---
% scrartcl ist Teil von KOMA-Script, sehr gut für deutsche Texte
\documentclass{scrartcl}

% --- Wichtige Pakete ---
% Für die korrekte Silbentrennung und deutsche Begriffe (z.B. "Abbildung")
\usepackage[ngerman]{babel}
% Für die richtige Darstellung von Umlauten im PDF (wichtig!)
\usepackage[T1]{fontenc}
% Definiert die Zeichenkodierung der .tex-Datei als UTF-8 (Standard heute)
\usepackage[utf8]{inputenc}
% Für mathematische Formeln
\usepackage{amsmath}

% --- Metadaten für das Dokument ---
\title{Ein kleiner Test für meine Bachelorarbeit}
\author{Sofia Schepers}
\date{\today} % \today fügt das aktuelle Datum ein


% --- Beginn des eigentlichen Dokuments ---
\begin{document}

% Titelseite erstellen
    \maketitle

% Ein erster Abschnitt


    \section{Einleitung}

    Hallo Welt! Dies ist mein erstes LaTeX-Dokument. Es ist gar nicht so schwer.
    Man kann hier einfach Text schreiben und Umlaute wie ä, ö, ü und das ß problemlos verwenden.

    Ein neuer Absatz wird durch eine Leerzeile im Code erzeugt.

    \subsection{Ein paar Elemente}
    Man kann auch ganz einfach Listen erstellen:
    \begin{itemize}
        \item Das ist der erste Punkt.
        \item Und hier der zweite.
        \item Man kann auch Formeln wie die berühmte Formel von Einstein, $E=mc^2$, direkt im Text einfügen.
    \end{itemize}

    Oder eine nummerierte Liste:
    \begin{enumerate}
        \item Erstens
        \item Zweitens
    \end{enumerate}

% Ein zweiter Abschnitt


    \section{Fazit}
    Das Kompilieren hat geklappt. Jetzt kann die richtige Arbeit beginnen!

% --- Ende des Dokuments ---
\end{document}