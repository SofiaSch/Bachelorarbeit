\section{Einleitung}
% Die Einleitung folgt dem "umgedrehten Pyramiden"-Prinzip[cite: 2484].
% 1. Relevanz des Themas (Praxis und/oder Wissenschaft)[cite: 2427].
% 2. Forschungsbedarf aufzeigen (Lücke, widersprüchliche Ergebnisse, methodische Mängel)[cite: 2430].
% 3. Ziel der Arbeit klar formulieren[cite: 2432].
% 4. Theoretischer Beitrag (Welche Wissenslücke wird geschlossen?)[cite: 2433].
% 5. Praktische Implikationen (Für wen sind die Ergebnisse relevant?)[cite: 2437].
% 6. Kurzer Ausblick auf die Vorgehensweise[cite: 2439].