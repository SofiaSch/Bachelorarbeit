\section{Einleitung}\label{sec:einleitung}

Die öffentliche Auftragsvergabe ist mit einem Volumen von rund 12 \% des Bruttoinlandsprodukts der Europäischen Union ein entscheidender Hebel für Wirtschaftswachstum und Innovation~\parencite{Value_Creation_in_PP}.
Trotz dieser Bedeutung stehen öffentliche Ausschreibungen häufig in der Kritik, durch übermäßige bürokratische Hürden insbesondere kleine und mittlere Unternehmen (KMU) vom Wettbewerb auszuschließen.
Während die Literatur bereits verschiedene Barrieren identifiziert hat, bleibt ein Faktor in seiner Wirkung oft ambivalent: die zeitliche Gestaltung der Verfahren.

Die Verfahrensdauer wird traditionell im Rahmen der Transaktionskostentheorie als Belastungsfaktor betrachtet~\parencite{transaktionskostentheorie}.
Lange Fristen binden Ressourcen und erhöhen die Unsicherheit für Bieter.
Dennoch ist unklar, ob Zeit in jedem Kontext als Abschreckung wirkt oder ob sie, gerade für ressourcenarme KMU, als notwendiger Raum für die Angebotserstellung dient~\parencite{buerokratie_DE}.
Diese Unklarheit wird durch die Heterogenität der europäischen Verwaltungslandschaft verstärkt.
Während Estland als digitaler Vorreiter auf maximale Geschwindigkeit setzt, sind Systeme wie jene in Deutschland und Frankreich durch tief verwurzelte föderale oder zentralistische Strukturen geprägt, die eine andere zeitliche Dynamik suggerieren.

Hier setzt die vorliegende Arbeit an.
Es fehlt bisher an einer vergleichenden empirischen Untersuchung, die analysiert, wie die geplante Verfahrensdauer die Beteiligung am Wettbewerb in Abhängigkeit vom nationalen Verwaltungskontext beeinflusst.
Daraus leitet sich die zentrale dieser Arbeit ab:

\begin{quote}
\textit{Inwiefern beeinflusst die geplante Verfahrensdauer die Wettbewerbsintensität und die KMU-Beteiligung in der öffentlichen Auftragsvergabe, und welche moderierende Rolle spielt dabei der länderspezifische Verwaltungskontext in Deutschland, Frankreich und Estland?}
\end{quote}

Zur Beantwortung dieser Frage wird eine quantitative Analyse auf Basis von Daten der Plattform OpenTender.eu durchgeführt.
Mithilfe von Regressionsmodellen (ZINB und Fractional Logit) wird untersucht, ob die Verfahrensdauer als universeller Kostentreiber oder als kontextabhängiges Signal fungiert.

Die Arbeit ist wie folgt strukturiert: Kapitel 2 legt die theoretischen Grundlagen der Trans\-aktions\-kosten- und Signaltheorie dar und beleuchtet die länderspezifischen Kontexte.
In Kapitel 3 werden die Hypothesen hergeleitet.
Kapitel 4 beschreibt das methodische Vorgehen und die Datengrundlage, während Kapitel 5 die empirischen Ergebnisse präsentiert.
Diese werden in Kapitel 6 vor dem Hintergrund der Theorie diskutiert und um praktische Implikationen ergänzt.
Kapitel 7 schließt die Arbeit mit einem Fazit ab.