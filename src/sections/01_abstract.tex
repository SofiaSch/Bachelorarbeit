\section*{Abstrakt}

Die Beteiligung von Unternehmen und insbesondere kleiner und mittlerer Unternehmen (KMU) an der öffentlichen Auftragsvergabe ist essenziell für den Wettbewerb in der EU.
Wird jedoch oft durch administrative Komplexität erschwert.
Vor diesem Hintergrund untersucht diese Bachelorarbeit den Einfluss der geplanten Verfahrensdauer auf die Wettbewerbsintensität und die KMU-Beteiligung unter Berücksichtigung des moderierenden nationalen Verwaltungskontexts.

Theoretisch fundiert durch die Transaktionskosten- und Signaltheorie dient die Verfahrensdauer als Indikator für die administrative Effizienz.
Die empirische Prüfung erfolgt durch eine quantitative Analyse von OpenTender.eu-Daten aus Deutschland, Frankreich und Estland.
Hierfür werden Zero-Inflated Negative Binomial- und Fractional-Logit-Modelle verwendet.

Die Ergebnisse widerlegen die Annahme, dass eine lange Verfahrensdauer grundsätzlich abschreckend wirkt.
In Deutschland und Frankreich zeigt sich sogar ein gegenteiliger Effekt: Hier führen längere Fristen zu einer höheren Anzahl an Geboten.
Dies deutet darauf hin, dass Unternehmen die zusätzliche Zeit als notwendige Ressource nutzen, um bessere Angebote zu erstellen.
In Estland hingegen bestätigt sich ein negativer Einfluss: Längere Verfahren schrecken dort insbesondere kleine und mittlere Unternehmen (KMU) ab, da sie als Signal für ineffiziente oder fehlerhafte Verwaltungsprozesse gedeutet werden.

Die Arbeit zeigt die Relevanz von Kontextfaktoren für die Transaktionskostentheorie auf.
Für die Praxis empfiehlt sich eine länderspezifische Steuerung der Fristen sowie eine stärkere Standardisierung und Zentralisierung der Vergabeprozesse, um die Verlässlichkeit administrativer Signale und den Marktzugang für KMU zu verbessern.

\begin{quote}
\textit{Kennwörter: Öffentliche Auftragsvergabe, KMU-Beteiligung, Verfahrensdauer, Transaktionskostentheorie, Signaltheorie, internationaler Vergleich.}
\end{quote}