\section*{Abstrakt}

Die Beteiligung kleiner und mittlerer Unternehmen (KMU) an der öffentlichen Auftragsvergabe ist essenziell für den Wettbewerb in der EU, wird jedoch oft durch administrative Komplexität erschwert.
Vor diesem Hintergrund untersucht diese Bachelorarbeit den Einfluss der geplanten Verfahrensdauer auf die Wettbewerbsintensität und die KMU-Beteiligung unter Berücksichtigung des moderierenden nationalen Verwaltungskontexts.

Theoretisch fundiert durch die Transaktionskosten- und Signaltheorie, dient die Verfahrensdauer als Indikator für die administrative Effizienz betrachtet.
Die empirische Prüfung erfolgt mittels einer quantitativen Analyse von OpenTender.eu-Daten aus Deutschland, Frankreich und Estland.
Hierfür werden Zero-Inflated Negative Binomial- und Fractional-Logit-Modelle verwendet.

Die Analyseergebnisse zeigen zunächst, dass eine universell abschreckende Wirkung der Verfahrensdauer (H1) widerlegt werden kann.
Konkret korrelieren längere Fristen in Deutschland und Frankreich tendenziell als notwendige Vorbereitungsressource positiv mit der Gebotsanzahl während sich in Estland ein signifikanter negativer Effekt auf die KMU-Beteiligung (H2) zeigt.
Hier signalisiert eine längere Dauer administrative Anomalien, die KMU überproportional belasten.

Die Arbeit zeigt die Relevanz von Kontextfaktoren für die Transaktionskostentheorie auf.
Für die Praxis empfiehlt sich eine länderspezifische Steuerung der Fristen sowie eine stärkere Standardisierung und Zentralisierung der Vergabeprozesse, um die Verlässlichkeit administrativer Signale und den Marktzugang für KMU zu verbessern.

\begin{quote}
\textit{Kennwörter: Öffentliche Auftragsvergabe, KMU-Beteiligung, Verfahrensdauer, Transaktionskostentheorie, Signaltheorie, internationaler Vergleich.}
\end{quote}