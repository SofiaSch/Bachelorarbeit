\newpage
\section{Ergebnisse}\label{sec:ergebnisse}

Die empirische Analyse der Vergabedaten fördert Muster zutage, die teilweise intuitiv erscheinen, an entscheidenden Stellen jedoch gängigen Annahmen widersprechen.
Um diese Dynamiken zu verstehen, bildet eine detaillierte Betrachtung der länderspezifischen Ausgangslage das Fundament, auf dem anschließend die statistische Überprüfung der Hypothesen mittels Regressionsmodellen aufbaut.

\subsection{Deskriptive Statistiken}\label{subsec:deskriptive-statistiken}

Die deskriptive Analyse der finalen Stichprobe ($N=150.140$) offenbart deutliche strukturelle Unterschiede zwischen den drei untersuchten Ländern.
Tabelle~\ref{tab:descriptive} fasst die zentralen Kennzahlen zusammen.

Hinsichtlich der unabhängigen Variable, der Verfahrensdauer, zeigt sich, dass Deutschland mit durchschnittlich 42,69 Tagen (Median: 32) die längsten Fristen ansetzt, gefolgt von Estland (Mittelwert: 39,67; Median: 33).
Frankreich weist mit durchschnittlich 38,33 Tagen (Median: 32) die kürzesten Fristen auf, verfügt jedoch über eine geringere Standardabweichung (32,53) als Deutschland (46,79).

Auffällig ist der hohe Anteil an Ausschreibungen ohne Gebotsabgabe (Zero-Inflation).
In Frankreich erhielten 96,21\,\% der Ausschreibungen kein valides Gebot, in Estland 78,94\,\% und in Deutschland 71,96\,\%.
Dies unterstreicht die Notwendigkeit des Zero-Inflated-Modells.
Bei den erfolgreichen Ausschreibungen (Gebote $>0$) ist der Wettbewerb in Deutschland am intensivsten (Mittelwert: 1,13 Gebote), während Frankreich (0,13) und Estland (0,65) deutlich niedriger liegen.

Der Anteil der KMU an den Bietern ist in Estland mit durchschnittlich 87\,\% am höchsten, gefolgt von Deutschland (64\,\%) und Frankreich (29\,\%).

\begin{table}[htbp]
    \centering
    \caption{Deskriptive Statistiken der Stichprobe nach Ländern}
    \label{tab:descriptive}
    \begin{tabular*}{\textwidth}{l@{\extracolsep{\fill}}rrr}
        \toprule
        \textbf{Variable} & \textbf{Estland} & \textbf{Frankreich} & \textbf{Deutschland} \\
        \midrule
        \textit{Verfahrensdauer (Tage)} & & & \\
        Mittelwert (SD) & 39,67 (37,07) & 38,33 (32,53) & 42,69 (46,79) \\
        Median & 33,00 & 32,00 & 32,00 \\
        \midrule
        \textit{Wettbewerb (Anzahl Gebote)} & & & \\
        Mittelwert (SD) & 0,65 (1,62) & 0,13 (1,15) & 1,13 (3,14) \\
        Anteil ohne Gebote & 78,94\,\% & 96,21\,\% & 71,96\,\% \\
        \midrule
        \textit{KMU-Anteil (wenn Gebote $>0$)} & & & \\
        Mittelwert (SD) & 0,87 (0,28) & 0,29 (0,43) & 0,64 (0,45) \\
        \midrule
        \textit{Auftragswert (Median in \euro)} & 520.000 & 560.000 & 505.530 \\
        \bottomrule
    \end{tabular*}
    \vspace{0.1cm}
    \parbox{\textwidth}{\footnotesize{\textit{Anmerkung:} SD = Standardabweichung. Datenbasis OpenTender 2014--2022. Werte für KMU-Anteil basieren nur auf Ausschreibungen mit mindestens einem Gebot.}}
\end{table}

Zur weiteren Veranschaulichung der zentralen unabhängigen Variable zeigt Abbildung~\ref{fig:boxplot_dauer} die Verteilung der Fristen in den drei Ländern.
Die Boxplots verdeutlichen, dass die Kernverteilung (Median und Interquartilsabstand) in allen Ländern ähnlich gelagert ist, Deutschland jedoch eine tendenziell größere Streuung und mehr Ausreißer im oberen Bereich aufweist.

\begin{figure}[H]
    \centering
    \includegraphics[width=0.8\textwidth]{images/01_boxplot_dauer}
    \caption{Verteilung der Verfahrensdauer (in Tagen) nach Ländern}
    \label{fig:boxplot_dauer}
    \vspace{0.2cm}
    \footnotesize{\textit{Quelle:} Eigene Darstellung basierend auf OpenTender-Daten (Python).}
\end{figure}

\subsection{Hypothesentests}\label{subsec:hypothesentests}

\subsubsection{Einfluss der Verfahrensdauer auf die Wettbewerbsintensität}
Zur Überprüfung von Hypothese 1 (H1) und Hypothese 3 (H3) wurde ein Zero-Inflated Negative Binomial (ZINB) Modell berechnet. Die Ergebnisse sind in Tabelle \ref{tab:zinb} dargestellt.

\begin{table}[htbp]
    \centering
    \caption{Ergebnisse der ZINB-Regression auf die Anzahl der Gebote (Wettbewerbsintensität)}
    \label{tab:zinb}
    \begin{tabular*}{\textwidth}{l@{\extracolsep{\fill}}rrr}
        \toprule
        \textbf{Variable} & \textbf{Koeffizient} & \textbf{Std. Fehler} & \textbf{z-Wert} \\
        \midrule
        \multicolumn{4}{l}{\textit{Count Model (Negative Binomial)}} \\
        Intercept (Referenz: Estland) & $-3,042^{***}$ & 0,093 & -32,82 \\
        Land: Frankreich & $-1,741^{***}$ & 0,059 & -29,62 \\
        Land: Deutschland & $0,203^{***}$ & 0,059 & 3,42 \\
        \textbf{Verfahrensdauer (z-std.)} & $\mathbf{-0,026}$ & \textbf{0,063} & \textbf{-0,41} \\
        \textbf{Dauer $\times$ Frankreich} & $\mathbf{0,148^{*}}$ & \textbf{0,066} & \textbf{2,25} \\
        \textbf{Dauer $\times$ Deutschland} & $\mathbf{0,105}$ & \textbf{0,064} & \textbf{1,65} \\
        Auftragswert (log, z-std.) & $0,068^{***}$ & 0,010 & 6,75 \\
        \midrule
        \multicolumn{4}{l}{\textit{Kontrollvariablen}} \\
        Verfahrensart (Selektiv) & $-0,238^{***}$ & 0,028 & -8,54 \\
        Sektor (Dienstleistung) & $0,358^{***}$ & 0.023 & 15,55 \\
        Sektor (Bauleistung) & $0,654^{***}$ & 0,028 & 23,06 \\
        Jahres-Fixed-Effects & \multicolumn{3}{c}{Ja (signifikant)} \\
        \midrule
        \multicolumn{4}{l}{\textit{Modellgüte}} \\
        Log-Likelihood & \multicolumn{3}{c}{-95.818} \\
        Pseudo $R^2$ & \multicolumn{3}{c}{0,063} \\
        Beobachtungen ($N$) & \multicolumn{3}{c}{150.140} \\
        \bottomrule
    \end{tabular*}
    \vspace{0.1cm}
    \parbox{\textwidth}{\footnotesize{\textit{Anmerkung:} Referenzkategorie für Land ist Estland. Dauer und Wert sind z-standardisiert.\\$^{*} p < 0.05, ^{**} p < 0.01, ^{***} p < 0.001 $.}}
\end{table}

Die statistischen Ergebnisse aus Tabelle~\ref{tab:zinb} lassen sich grafisch verdeutlichen.
Abbildung~\ref{fig:interaction_wettbewerb} zeigt die vorhergesagten Werte (Predicted Margins) für die Anzahl der Gebote in Abhängigkeit von der Verfahrensdauer.
Es wird deutlich, dass die Kurve für Estland nahezu flach verläuft, was den nicht-signifikanten Effekt widerspiegelt.
Im Gegensatz dazu zeigen Deutschland und Frankreich eine sichtbare positive Steigung: Längere Verfahren gehen in diesen Ländern mit einer höheren Anzahl an Geboten einher.

\begin{figure}[H]
    \centering
    \includegraphics[width=0.75\textwidth, trim=0cm 0cm 0cm 0cm, clip]{images/02_interaction_wettbewerb_H1_H3a}
    \caption{Interaktionseffekt der Verfahrensdauer auf die vorhergesagte Anzahl an Geboten}
    \label{fig:interaction_wettbewerb}
    \vspace{0.2cm}
    \parbox{0.75\textwidth}{\footnotesize{\textit{Quelle:} Eigene Darstellung basierend auf ZINB-Modellschätzung (Python).}}
\end{figure}

Die Analyse zeigt Folgendes:
\begin{itemize}
    \item Für \textbf{Estland} (Referenz) ist der Effekt der Verfahrensdauer negativ ($-0,026$), aber statistisch nicht signifikant ($p=0,681$).
    Es besteht also kein nachweisbarer Zusammenhang zwischen Dauer und Anzahl der Gebote.
    \item Für \textbf{Frankreich} zeigt sich ein signifikant positiver Interaktionseffekt ($0,148^{*}$).
    Der Gesamteffekt ist somit positiv: Längere Verfahrensdauern führen hier tendenziell zu \textit{mehr} Geboten.
    \item Für \textbf{Deutschland} ist der Interaktionseffekt ebenfalls positiv ($0,105$), verfehlt jedoch knapp das 5\%-Signifikanzniveau ($p=0,100$).
    Der Trend deutet jedoch ebenfalls auf einen positiven Zusammenhang hin.
\end{itemize}
Hypothese 1 (länger = weniger Gebote) muss somit abgelehnt werden.
In den großen Volkswirtschaften scheint eine längere Frist eher als Chance für komplexere Angebote genutzt zu werden, statt abzuschrecken.

\subsubsection{Einfluss der Verfahrensdauer auf die KMU-Beteiligung}
Zur Überprüfung von Hypothese 2 (H2) wurde ein Fractional Logit Modell (GLM) für die Subgruppe der erfolgreichen Ausschreibungen ($N=19.760$) berechnet (siehe Tabelle~\ref{tab:glm}).

\begin{table}[htbp]
    \centering
    \caption{Ergebnisse der Fractional Logit Regression auf den KMU-Anteil}
    \label{tab:glm}
    \begin{tabular*}{\textwidth}{l@{\extracolsep{\fill}}rrr}
        \toprule
        \textbf{Variable} & \textbf{Koeffizient} & \textbf{Std. Fehler} & \textbf{z-Wert} \\
        \midrule
        Intercept (Referenz: Estland) & $-2,317^{***}$ & 0,643 & -3,61 \\
        Land: Frankreich & $-2,981^{***}$ & 0.104 & -28,67 \\
        Land: Deutschland & $-2,014^{***}$ & 0,098 & -20,57 \\
        \textbf{Verfahrensdauer (z-std.)} & $\mathbf{-0,277^{***}}$ & \textbf{0,073} & \textbf{-3,79} \\
        \textbf{Dauer $\times$ Frankreich} & $\mathbf{0,295^{***}}$ & \textbf{0,090} & \textbf{3,26} \\
        \textbf{Dauer $\times$ Deutschland} & $\mathbf{0,285^{***}}$ & \textbf{0,074} & \textbf{3,85} \\
        Auftragswert (log, z-std.) & $-0,139^{***}$ & 0,015 & -9,05 \\
        \midrule
        \multicolumn{4}{l}{\textit{Kontrollvariablen}} \\
        Verfahrensart (Selektiv) & $0,633^{***}$ & 0,044 & 14,48 \\
        Sektor (Dienstleistung) & $0,604^{***}$ & 0,043 & 14,19 \\
        Sektor (Bauleistung) & $1,125^{***}$ & 0,041 & 27,23 \\
        Jahres-Fixed-Effects & \multicolumn{3}{c}{Ja (signifikant)} \\
        \midrule
        \multicolumn{4}{l}{\textit{Modellgüte}} \\
        Log-Likelihood & \multicolumn{3}{c}{-11.222} \\
        Beobachtungen ($N$) & \multicolumn{3}{c}{19.760} \\
        \bottomrule
    \end{tabular*}
    \vspace{0.2cm}
    \parbox{\textwidth}{\footnotesize{\textit{Anmerkung:} Referenzkategorie für Land ist Estland.\\$^{*} p < 0.05, ^{**} p < 0.01, ^{***} p < 0.001 $.}}
\end{table}

Die in Tabelle~\ref{tab:glm} ausgewiesenen Interaktionseffekte werden in Abbildung~\ref{fig:interaction_kmu} visualisiert.
Die Grafik illustriert die vorhergesagte Wahrscheinlichkeit für den KMU-Anteil an den Bietern.
Hier zeigt sich der drastische Unterschied zwischen den Systemen: Während die Linien für Deutschland und Frankreich stabil verlaufen, fällt die Kurve für Estland steil ab.
Dies verdeutlicht visuell, dass Zeitverzögerungen im estnischen System eine massive Barriere speziell für kleine und mittlere Unternehmen darstellen.

\begin{figure}[H]
    \centering
    \includegraphics[width=0.75\textwidth, trim=0cm 0cm 0cm 0cm, clip]{images/03_interaction_kmu_H2_H3b}
    \caption{Interaktionseffekt der Verfahrensdauer auf den Anteil der KMU-Gebote}
    \label{fig:interaction_kmu}
    \vspace{0.2cm}
    \footnotesize{\textit{Quelle:} Eigene Darstellung basierend auf Fractional-Logit-Modellschätzung (Python).}
\end{figure}

Die Ergebnisse stützen Hypothese 2 differenziert nach Ländern:
\begin{itemize}
    \item In \textbf{Estland} zeigt sich ein hochsignifikanter negativer Effekt ($\beta = -0,277, p < 0,001$).
    Je länger das Verfahren dauert, desto geringer ist der Anteil der KMU an den Bietern.
    Hier bestätigt sich die Annahme, dass Zeitverzögerungen als Ressourcenbelastung für kleinere Unternehmen wirken.
    \item In \textbf{Deutschland} und \textbf{Frankreich} hingegen neutralisieren die signifikant positiven Interaktionseffekte ($0,285^{***}$ bzw. $0,295^{***}$) diesen negativen Basiseffekt fast vollständig.
    Der Gesamteffekt liegt nahe Null.
\end{itemize}
Zusammenfassend bestätigt sich H2 nur für das hocheffiziente System Estlands, während in Deutschland und Frankreich keine Benachteiligung der KMU durch längere Fristen nachweisbar ist.
Auch Hypothese 3 (Deutschland reagiert am stärksten negativ) muss daher abgelehnt werden; tatsächlich reagiert Estland am sensibelsten.

Diese unerwarteten Befunde – insbesondere die gegensätzlichen Effekte in Estland im Vergleich zu den großen Volkswirtschaften werfen fundamentale Fragen zur Wahrnehmung von Verfahrensdauer in unterschiedlichen Verwaltungssystemen auf, die im folgenden Kapitel diskutiert werden.