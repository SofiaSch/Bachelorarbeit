\newpage
\section{Fazit}\label{sec:fazit}

Die vorliegende Bachelorarbeit untersuchte den Einfluss der geplanten Verfahrensdauer auf die Wettbewerbsintensität und die KMU-Beteiligung in der öffentlichen Auftragsvergabe in Deutschland, Frankreich und Estland.
Ziel war es, die klassische Annahme der Transaktionskostentheorie, dass Zeit primär als Kostenfaktor und Barriere wirkt, im Kontext unterschiedlicher administrativer Strukturen kritisch zu hinterfragen.

Die empirische Analyse auf Basis von Daten von Open Tender liefert ein differenziertes Ergebnis: Die zentrale Forschungsfrage nach dem Einfluss der Verfahrensdauer lässt sich nicht universell beantworten, sondern hängt maßgeblich vom institutionellen Rahmen des jeweiligen Landes ab.
Während in Deutschland und Frankreich eine längere Dauer tendenziell sogar mit einer höheren Bieterzahl korreliert, zeigt sich in Estland ein deutlicher Abschreckungseffekt auf KMU.

Daraus lässt sich das wesentliche Fazit dieser Arbeit ableiten: Die Verfahrensdauer fungiert als institutionell moderiertes Signal.
In einem hoch digitalisierten und effizienten Umfeld wie Estland wird Zeitverzögerung als administrative Anomalie und Risiko interpretiert.
In den komplexeren Systemen Deutschlands und Frankreichs hingegen wird sie eher als notwendige Ressource für die Angebotserstellung wahrgenommen.

Für die Praxis bedeutet dies, dass eine pauschale Verkürzung von Vergabefristen nicht zwingend zu mehr Wettbewerb führt.
Vielmehr müssen politische Entscheidungsträger die Signalwirkung ihrer Prozesse verstehen.
Während in Deutschland die Zentralisierung und Standardisierung zur Entlastung der Fachkräfte und zur Erhöhung der Vorhersehbarkeit im Vordergrund stehen sollte, ist in digitalen Vorreitersystemen die strikte Einhaltung von Zeitplänen essentiell, um KMU nicht systematisch zu benachteiligen.

Zukünftige Forschung sollte die hier identifizierten Signalwirkungen durch qualitative Befragungen vertiefen, um den unerforschten Aspekt der unternehmerischen Entscheidungsfindung in Abhängigkeit von Zeitparametern weiter zu entschlüsseln.
Letztlich zeigt die Arbeit, dass Bürokratieabbau nicht nur eine Frage der Geschwindigkeit, sondern vor allem der Verlässlichkeit und Transparenz administrativer Signale ist.

