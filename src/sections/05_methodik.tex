\newpage
\section{Methodik}\label{sec:methodik}

Um die Hypothesen zu überprüfen und somit die Forschungsfrage zu beantworten führen wir ein quantitatives, komparatives Ex-post-facto-Forschungsdesign durch.
Die Hypothese wird mittels statistischer Regressionsanalysen auf Basis numerischer Daten getestet und ist somit quantitativ.
Der gezielte Vergleich der drei Länder Deutschland, Frankreich und Estland nach dem \("\)Most Different System Design\("\) (MDSD) stellt eine komparative Analyse dar.
Zudem beziehen wir uns in der Analyse auf bereits vergangene Vergabeverfahren die nicht experimentell manipuliert wurden, was das Design als Ex-post-facto-Ansatz ausweist.

\subsection{Datensatz und Stichprobe}\label{subsec:datensatz-und-stichprobe}

\subsubsection{Datenquelle und -gewinnung}

Die Daten, die für die empirische Analyse genutzt werden stammen aus der OpenTender Plattform, welche vom Government Transparency Institute gesammelt, transformiert und veröffentlicht wird~\parencite{GTI2024}.
In der Datenbank befindet sich die Daten von insgesamt 35 Gerichtsbarkeiten, davon 27 Mitgliedstaaten der Europäischen Union~\parencite{GTI2024}.
Es handelt sich um einen Sekundärdatensatz.

Die Daten werden unter anderem im JSON Format zur Verfügung gestellt.
Der Untersuchungszeitraum ist auf die Jahre von 2014 bis 2022 festgelegt.
In 2014 wurde ein großer Schritt bezüglich des europäischen Vergaberechts gemacht, da dort neue EU-Vergaberichtlinien verabschiedet wurden.
Diese zielen auf eine Vereinfachung der Verfahren und Stärkung der KMU-Beteiligung ab~\parencite{EU2014RL24}.
Die Wahl für das Jahr 2022 beruht auf der Datenverfügbarkeit, da das neusten Daten für Frankreich aus dem Jahre 2022 sind.
Die Analyse dieses Zeitraums ermöglicht die Analyse innerhalb des aktuellen rechtlichen Rahmens auf einer gemeinsamen Grundlage.

\subsubsection{Länderauswahl}

Die Auswahl der Länder basiert auf dem "Most Different Systems Design" (MDSD).
Ziel ist es dabei drei Länder, innerhalb des gemeinsamen rechtlichen Rahmens der EU, mit unterschiedlichen Verwaltungsstrukturen zu vergleichen.

Deutschland zählt als Vorreiter in der Umsetzung der neuen EU-Vergaberichtlinien und ist zudem hochgradig föderalistisch und regulierungsintensiv~\parencite{paper1}.
Die dezentrale Vergabe erfolgt auf Bundes-, Landes- und Kommunalebene, was zu einer heterogenen Verwaltungslandschaft führt~\parencite{DE_FOER}. \n
Frankreich dient als gegensatz mit einem historisch zentralem Staat~\parencite{FR_Zentr}.
Die Vergaben sind größtenteils landesweit standardisiert
Als drittes Land ist Estland deutlich kleiner als die beiden wirtschaftlich starken Länder Deutschland und Frankreich.
Dennoch gilt Estland als digitaler Vorreiter, was die Digitalisierung in Europa betrifft~\parencite{EST_Digital}.

Durch den gezielten Vergleich dieser unterschiedlichen Systeme wird es möglich, über eine reine Effektmessung hinauszugehen und zu einem tieferen, kausalen Verständnis darüber zu gelangen, warum und unter welchen Bedingungen die Dauer von Vergabeverfahren den Wettbewerb tatsächlich beeinflusst.

\subsubsection{Stichprobenziehung}

Die Analyse der Daten erfordert zunächst deren Umwandlung in ein geeignetes Format, welches die Datenaufbereitung für die nachfolgende Analyse erleichtert.
Mittels eines Python-Skripts erfolgte eine Analyse aller Einträge pro Land, bei der ausschließlich die Variablen extrahiert wurden, die für die Forschungsfrage von Relevanz sind.
Im Rahmen der weiteren Analyse erfolgte eine Konvertierung der Daten in eine CSV-Datei pro Land.
Daraufhin folgt die Bereinigung.

Im ersten Schritt der Bereinigung wurde die Datumsspalte von einem Text in ein Datums-Format konvertiert.
Nachfolgend wurden sämtliche Einträge eliminiert, bei denen das Start- oder Enddatum nicht angegeben war, da diese für die Analyse nicht relevant waren.
Bei einigen Einträgen treten Logikfehler auf, die sich dadurch äußern, dass das Enddatum vor dem Startdatum liegt.
Auch diese wurden aus den betreffenden Dateien entfernt.
Anschließend wurde eine Filterung der Einträge vorgenommen, die auch im wettbewerblichen Verfahren partizipiert haben, um eine Ausschließung von Direktvergaben zu gewährleisten.

Durch die Implementierung dieser Schritte wurde eine Reduktion der Datenmenge von 1.547.352 Einträgen (Deutschland: 402.358, Frankreich: 1.097.001, Estland: 47.993) auf 675.400 Einträge (Deutschland: 303.770, Frankreich: 342.021, Estland: 29.609) durchgeführt, um eine konsistente und signifikante Stichprobe zu erhalten.

\subsection{Variablen und Operationalisierung}\label{subsec:variablen-und-operationalisierung}
Als unabhängige Variable wird die geplante Dauer des Ausschreibungszeitraums in Tagen bemessen.
Dabei wird die Differenz zwischen dem Datum der geplanten Angebotsabgabe (endDate) und dem Datum der Veröffentlichung der Ausschreibung (publicationDate) berechnet.

Für unsere Analyse brauchen wir zwei abhängige Variablen.
Die erste beschreibt die Wettbewerbsintensität und wird direkt durch die absolute Anzahl der eingegangenen Gebote (total\_bids) pro Ausschreibung gemessen.
Die zweite abhängige Variable beschreibt die KMU-Beteiligung.
Diese wird nicht, wie bei der Wettbewerbsintensität in absoluten Zahlen gemessen sondern als prozentualer Anteil der Gesamtgebote.
Dafür rechnen wir Anzahl der KMU-Gebote (sme\_bids) durch die Gesamtzahl der Gebote (total\_bids).
Dadurch, dass wir hier den Anteil berechnen und nicht die Absolute Zahl können wir messen, ob Kleine und Mittelständige Unternehmen tatsächlich überproportional von längeren Verfahrensdauern betroffen sind oder nicht.

Für eine Robuste Analyse führen wir weitere Analysen durch mit ausgewählten Kontrollvariablen.
Zum einen analysieren wir, ob der Auftragswert (tender\_value)

\textcolor{red}{
Um den Netto-Effekt der Verfahrensdauer auf die Wettbewerbsintensität zu isolieren und Verzerrungen durch ausgelassene Variablen (Omitted Variable Bias) zu minimieren, werden in den Regressionsmodellen mehrere Kontrollvariablen berücksichtigt.
Diese Variablen wurden auf Basis der Fachliteratur zur öffentlichen Auftragsvergabe ausgewählt, da anzunehmen ist, dass sie sowohl mit der Dauer des Ausschreibungszeitraums als auch mit der Anzahl der eingegangenen Gebote korrelieren.
\begin{itemize}
\item Geschätzter Auftragswert (tender\_value): Der finanzielle Umfang einer Ausschreibung ist eine der wichtigsten Determinanten für das Bieterverhalten. Es ist davon auszugehen, dass Aufträge mit einem höheren Wert längere Angebotsfristen erhalten, um komplexere Angebote zu ermöglichen. Gleichzeitig beeinflusst der Wert die Attraktivität der Ausschreibung und damit die Anzahl der Bieter. Um eine Scheinkorrelation zwischen Dauer und Bieteranzahl zu vermeiden, wird der logarithmierte Auftragswert in das Modell aufgenommen.
\item Verfahrensart (procurement\_method): Die Art des Vergabeverfahrens hat einen direkten strukturellen Einfluss auf den Wettbewerb. Offene Verfahren (open) sind definitionsgemäß für eine unbegrenzte Anzahl von Unternehmen zugänglich und weisen oft andere Fristen auf als selektive Verfahren (selective), bei denen der Bieterkreis eingeschränkt ist. Die Kontrolle für die Verfahrensart ist daher notwendig, um diese institutionellen Rahmenbedingungen zu berücksichtigen.
\item Art der Leistung (procurement\_category): Die Wettbewerbsdynamik unterscheidet sich erheblich zwischen der Vergabe von Lieferleistungen (goods), Dienstleistungen (services) und Bauleistungen (works). Jeder dieser Sektoren hat eine eigene Marktstruktur, Anbieterdichte und typische Komplexität, was sich sowohl auf die übliche Dauer von Ausschreibungen als auch auf die zu erwartende Anzahl von Bietern auswirkt.
\item Jahr der Ausschreibung (year): Um für zeitliche Trends wie makroökonomische Zyklen, technologische Entwicklungen oder die schrittweise Wirkung von Gesetzesreformen (z.B. der EU-Vergaberichtlinien von 2014) zu kontrollieren, wird das Jahr der Veröffentlichung als kategoriale Variable in die Analyse einbezogen.
\item Land (country): Neben der Untersuchung von länderspezifischen Effekten durch Interaktionsterme dient die Ländervariable als Kontrollvariable für grundlegende, zeitinvariante Unterschiede zwischen den nationalen Beschaffungsmärkten. Faktoren wie die Wirtschaftsgröße, die Anzahl potenzieller Bieter oder die administrative Kultur können zu systematisch unterschiedlichen Wettbewerbsniveaus in Deutschland, Frankreich und Estland führen.
\item Vergabekriterien (award\_criteria): Die Kriterien für den Zuschlag beeinflussen den Aufwand für die Angebotserstellung. Eine Vergabe, die ausschließlich auf dem niedrigsten Preis basiert (priceOnly), ist für Bieter einfacher zu kalkulieren als eine, die auf dem wirtschaftlich vorteilhaftesten Angebot mit bewerteten Kriterien (ratedCriteria) beruht. Letztere erfordert komplexere Angebote, was längere Fristen rechtfertigen und gleichzeitig die Bieterzahl beeinflussen kann. Diese Variable wird daher in einer Robustheitsanalyse berücksichtigt, um die Stabilität der Ergebnisse zu prüfen.
\end{itemize}
}

\subsection{Statistische Auswertungsmethode}\label{subsec:statistische-auswertungsmethode}
% Welche statistischen Verfahren wurden genutzt? (z.B. multiple Regressionsanalyse).