\newpage
\section{Theoretischer Hintergrund}\label{sec:theoretischer-hintergrund}

Die Beantwortung der zentralen Forschungsfrage, welche Auswirkungen die Verfahrensdauer auf die Wettbewerbsintensität und die Beteiligung von KMUs hat, bildet den Gegenstand dieses Kapitels.
Zu diesem Zweck wird zunächst der theoretische Hintergrund erörtert.
In der vorliegenden Arbeit werden nach einer Darlegung der allgemeinen Informationen über das Vergabeverfahren die spezifischen Aspekte des Themas beleuchtet, woraus sich schließlich die zu prüfenden Hypothesen ergeben.

\subsection{Die Grundlagen der öffentlichen Auftragsvergabe}\label{subsec:die-grundlagen-der-offentlichen-auftragsvergabe}

Die öffentliche Auftragsvergabe stellt einen der bedeutendsten Wirtschaftsfaktoren moderner Volkswirtschaften dar.
Mit einem Anteil von etwa 12\% am Bruttoinlandsprodukt (BIP) und fast einem Drittel der gesamten Staatsausgaben in den OECD-Ländern werden immense Summen in Infrastruktur, Güter und Dienstleistungen investiert~\parencite{Value_Creation_in_PP}.
Allein in Deutschland flossen im Jahr 2019 rund 35\% der Staatsausgaben über diesen Kanal in die Wirtschaft, wobei Länder und Kommunen die größten Auftraggeber sind~\parencite{Oeffentliche_Auftraege_DE}.

Doch was genau verbirgt sich hinter dem Begriff des öffentlichen Vergabeverfahrens?
Im Allgemeinen handelt es sich um einen Vertrag zwischen der öffentlichen Hand und Unternehmen über die Erbringung von Liefer-, Dienst- oder Bauleistungen.
Als öffentliche Auftragsgeber werden alle Dienstellen des Bundes, der Länder, Gemeindeverbände und sonstige juristische Personen des öffentlichen Rechts, wie beispielsweise Hochschulen, definiert.
Außerdem Einrichtungen, die vom Staat finanziert werden (z.B.\ Krankenhäuser) und weitere Besonderheiten~\parencite{Oeffentliche_Auftraege_DE}.

Im Jahr 2014 wurde ein großer Schritt bezüglich des europäischen Vergaberechts gemacht, da dort neue EU-Vergaberichtlinien verabschiedet wurden~\parencite{Oeffentliche_Auftraege_DE}.
Diese zielen auf eine Vereinfachung der Verfahren und Stärkung der KMU-Beteiligung ab~\parencite{EU2014RL24}.
Dadurch haben sich grundlegende Prinzipien für das Vergabeverfahren etabliert, die das Ziel verfolgen das objektiv beste Angebot mit einem optimalen Preis-Leistung-Verhältnis auszuwählen.
Besonders relevant, auch im Bezug auf die KMU, gelten die 3 Prinzipien: Gleichheit, Nichtdiskriminierung und Transparenz~\parencite{JIM_Fair_Transparent_Competitive}.
Sie sollen garantieren, dass alle Unternehmen die gleiche Chance haben und niemand bevorzugt wird.
Als zentrales Prinzip gilt der Grundsatz des Wettbewerbs, welcher dazu führt, dass alle anderen Prinzipien erreicht werden~\parencite{Oeffentliche_Auftraege_DE}.
Um zudem das ökonomische Ziel zu erfüllen, gilt das Prinzip der Effizienz beziehungsweise Wirtschaftlichkeit~\parencite{JIM_Fair_Transparent_Competitive}.
Dieses Prinzip gewährleistet, dass hinsichtlich der eingesetzten Steuergelder das wirtschaftlichste und finanziell beste Resultat erzielt wird.

Diese Prinzipien bilden die normative Grundlage, auf der das gesamte System der öffentlichen Auftragsvergabe in der EU beruht.

Ein weiteres wichtiges Instrument im Vergaberecht sind die Schwellenwerte.
Diese werden von der EU-Kommission festgelegt und definieren, ob ein Auftrag national oder europaweit ausgeschrieben werden muss.
Bei sehr geringen Auftragswerten ist keine Ausschreibung notwendig, um den bürokratischen Aufwand gering zu halten.
Die Höhe der Schwellenwerte unterscheidet sich je nach Art des Auftrags~\parencite{schwellenwerte}.

In der Auftragsvergabe gibt es mehrere Verfahrensarten.
Der Unterschied besteht im Wesentlichen in ihrer Offenheit und dem Grad des Wettbewerbs.
In der öffentlichen Ausschreibung (beziehungsweise dem offenen Verfahren) werden eine unbegrenzte Anzahl an Unternehmen bekannt gegeben.
Dadurch wird ein uneingeschränkter Wettbewerb garantiert und somit das wirtschaftlich wertvollste Angebot ermittelt.
Alternativ gibt es die beschränkte Ausschreibung (beziehungsweise das nicht offene Verfahren).
Die Anzahl der Unternehmen ist beschränkt, welche von dem Auftraggeber angesprochen werden.
Zusätzlich gibt es noch die Verhandlungsvergabe, wodurch die Auftraggeber den meisten Freiraum haben.
Hier wird direkt mit ausgewählten Unternehmen verhandelt, sie ist jedoch nur unter bestimmten Voraussetzungen zulässig.
Unabhängig von der gewählten Art folgt der Verfahrensablauf einem strukturierten, mehrstufigen Prozess~\parencite{Oeffentliche_Auftraege_DE}.

Der grobe Verfahrensablauf ist in mehrere Schritte unterteilt, wie Abbildung~\ref{fig:verfahrensablauf} veranschaulicht.
Im ersten Schritt erfolgt die Auftragsbekanntmachung.
Je nachdem, ob es sich um ein nationales oder ein europaweites Verfahren handelt, können die Unterlagen entweder direkt heruntergeladen oder angefordert werden.
Anschließend können die Teilnahmeanträge eingereicht und Angebote abgegeben werden.
Die Angebote werden geprüft und bewertet, anschließend wird die Zuschlagserteilung bekannt gegeben.

\begin{figure}[h!]
    \centering
    \includegraphics[width=0.8\textwidth]{images/verfahrensablauf}
    \caption{Der grobe Ablauf des öffentlichen Vergabeverfahrens.}
    \label{fig:verfahrensablauf}
    \footnotesize
    \parencite{Oeffentliche_Auftraege_DE}
\end{figure}

\subsection{Bürokratie als Transaktionskostentreiber}\label{subsec:burokratie-als-transaktionskostentreiber-in-der-offentlichen-vergabe}

Damit die Grundprinzipien wie Transparenz, Wettbewerb und Gleichbehandlung eingehalten werden können erfordert das formalisierte und standardisierte Prozesse.
Diese erfolgen in Regeln, Dokumentationspflichten und mehrstufigen Abläufen und bilden somit die Bürokratie des Vergabewesens.
Was zweifelsfrei notwendig ist, kann dennoch eine Hürde sein, da eine hohe Bürokratie mit hohen Kosten verbunden ist~\parencite{Discretion_supplier_selection}.
In diesem Fall werden die Kosten nicht zwangsläufig in monetären Werten berechnet, sondern in Form von Transaktionskosten.
Das sind alle Kosten, die beim Abschluss eines Geschäfts entstehen~\parencite{transaktionskosten}.
Darunter fallen die direkten Kosten, die sofort sichtbar und messbar sind.
Wie auch die indirekten Kosten, wie zum Beispiel die Such- und Informationskosten oder Verhandlungs- und Entscheidungskosten~\parencite{transaktionskosten_2}.
Zusammengefasst also alle Aufwände, die ein Unternehmen hat, um überhaupt an der Ausschreibung teilnehmen zu können.

In einer Ausschreibung fallen direkte Kosten in Form von Ressourcenbindung an.
Dazu gehört die Kapitalbindung, was insbesondere die Opportunitätskosten erhöht, da dieses Kapital nicht für andere Projekte genutzt werden kann~\parencite{opportunitaetskosten}.
Außerdem die Personalbindung, da sich wichtige Mitarbeiter mit der Ausschreibung befassen und ihre Kapazitäten in das ausarbeiten des Angebots stecken müssen.

Indirekte Kosten treten durch die Signalwirkung auf.
Eine lange Frist gilt oft als Signal für Komplexität, wodurch der Prozess als ineffizient und schwerfällig interpretiert wird.
Zudem kann eine lange Frist für Unsicherheit, sowohl auf der Angebots- als auch auf der Nachfrageseite - je länger der Prozess, desto größer das Risiko, dass sich Marktbedingungen, Kosten oder Anforderungen ändern~\parencite{PPP_transparency}.

Bürokratie verursacht Transaktionskosten, woraus sich die logische Konsequenz ergibt, dass bei steigenden Kosten und Risiken der erwartete Nettonutten für ein Unternehmen sinkt.
Dementsprechend werden sich Unternehmen gegen eine Angebotsabgabe von übermäßig langen Verfahren entscheiden~\parencite{Discretion_supplier_selection}.

Während diese steigenden Transaktionskosten alle Unternehmen betreffen, sind kleine und mittlere Unternehmen aufgrund ihrer spezifischen strukturellen Merkmale davon in besonderem Maße betroffen~\parencite{PPP_transparency}.

\subsection{Die Sondersituationen von kleinen und mittleren Unternehmen (KMU)}\label{subsec:die-sondersituationen-von-kleinen-und-mittleren-unternehmen-(kmu)}

Im Jahr 2023 zählten alleine 99\% aller Unternehmen in Deutschland zu der Gruppe der kleinen und mittleren Unternehmen.
Das umfasst ein Beschäftigungsvolumen von etwa 53\% und 41\% der Bruttowertschöpfung~\parencite{KMU_numbers}.
Die Zahlen alleine zeigen bereits, dass KMU ein großer und wichtiger Teil der Wirtschaft sind.
Und auch die Erleichterungen durch die Reform zeigen, dass der Staat die relevanz erkannt hat:

\begin{blockquote}
    Die öffentliche Vergabe sollte an die Bedürfnisse von KMU angepasst werden.~\parencite[Erwägungsgrund 78]{EU2014RL24}
\end{blockquote}

So sieht die neue Reform vor, dass große Aufträge in kleinere Teil- und Fachlose aufgeteilt wird, sodass diese auch von kleinen Unternehmen erfüllt werden können~\parencite[Artikel 46]{EU2014RL24}.
Eine andere Maßnahme ist, dass das Unternehmen keine unverhältnismäßig großen Jahresumsatz vorweisen muss, um an der Ausschreibung teilzunehmen.
Seit der Reform darf in der Regel nur noch das Zweifache des Auftragswertes gefordert werden~\parencite[Artikel 58, Absatz 3]{EU2014RL24}.

Obwohl die Richtlinien das Vergabeverfahren für KMU geöffnet hat, besteht weiterhin Hürden, die die Teilnahme erschwert.
Kapital und Personal ist begrenzt und besonders junge und sehr kleine Unternehmen verfügen über wenig Ressourcen, die sie für komplexe Ausschreibungen zur Verfügung haben.
Auch zu externem Kapital ist der Zugang meist begrenzt~\parencite{SME_Instruments}.
Das kann bedeuten, dass KMU weder die Zeit noch das Kapital haben um neben dem Tagesgeschäft weitere Ressourcen für Ausschreibungen zu entbehren.

Ein weiterer Punkt ist die geringere Erfahrung im Umgang mit übermäßig bürokratischen Prozessen~\parencite{JIM_Fair_Transparent_Competitive}.
Das führt zu erhöhter Unsicherheit sich überhaupt erst zu bewerben oder durch formale Fehler ausgeschlossen zu werden.

Diese Kombination von geringeren Ressourcen und erhöhten Ressourcen kann zur folge haben, dass bei einem langen Verfahren mit hohen Transaktionskosten der Abschreckungseffekt überproportional hoch ist.
Der Effekt tritt bei KMU stärker auf, trotz der neuen Reform, was dazu führt, das diese aus dem Wettbewerb verdrängt werden~\parencite{PPP_transparency}.




\subsection{Der Länderspezifische Verwaltungskontext als Moderator}\label{subsec:der-landerspezifische-verwaltungskontext-als-moderator}

Das öffentliche Vergabewesen in der Europäischen Union wird maßgeblich durch Richtlinien gerahmt, die eine weitgehende Harmonisierung der Verfahren für Aufträge oberhalb der EU-Schwellenwerte anstreben.
Da der für diese Analyse verwendete Datensatz primär solche oberschwelligen Verfahren umfasst, gelten für Deutschland, Frankreich und Estland theoretisch dieselben rechtlichen Rahmenbedingungen hinsichtlich Fristen und Bekanntmachungen.
Dennoch lassen sich signifikante Unterschiede in der Praxis der Auftragsvergabe erwarten.
Diese resultieren nicht aus unterschiedlichen Gesetzen, sondern aus der administrativen Umsetzung und der gelebten Verwaltungskultur innerhalb der Mitgliedsstaaten.
Nationale Unterschiede in der Behördenstruktur – etwa zentralisiert versus föderal – und dem Digitalisierungsgrad beeinflussen entscheidend, wie effizient die EU-Vorgaben in die Praxis umgesetzt werden.
Dies bestimmt maßgeblich, wie hoch die tatsächlichen Transaktionskosten für Bieter ausfallen und wie stark diese den Wettbewerb beeinträchtigen.


\subsubsection{Deutschland: Föderale Komplexität und Regelungsdichte}

Deutschlandweit existieren rund 30.000 Vergabestellen, die sich auf Bundes-, Landes- oder kommunaler Ebene befinden~\parencite{buerokratie_DE}.
Diese hohe Zahl an dezentralen Akteuren führt zu einer heterogenen Verwaltungslandschaft.
Je nach Standort sind die Vergabestellen unterschiedlich ausgestattet, was Finanzen und Personal betrifft, obwohl von ihnen erwartet wird, dass sie sich mit allen Arten von Beschaffungen auskennen.

Diese fragmentierte Struktur ist gerade für den hier untersuchten Oberschwellenbereich von zentraler Bedeutung.
Im Gegensatz zu zentralisierten Systemen obliegt die Durchführung komplexer EU-weiter Ausschreibungen in Deutschland weiterhin oft den dezentralen Bedarfsstellen vor Ort.
Dies führt zu einem strukturellen Defizit an Professionalisierung und Routine: Viele kleinere Vergabestellen müssen die hochkomplexen Anforderungen des EU-Vergaberechts umsetzen, ohne über die spezialisierten Fachkapazitäten zentraler Beschaffungsbehörden zu verfügen.~\parencite{buerokratie_DE}.

Das föderale System Deutschlands verstärkt diesen Effekt.
Es gilt das Grundprinzip der Zuständigkeit, wonach der Verwaltungsvollzug in der Regel Ländersache ist.
Dies berechtigt die Bundesländer eigene Organisations- und Verfahrensregeln aufzustellen, was unter anderem zu einer Zersplitterung der E-Vergabe-Landschaft führt.
Hinzu kommt das Recht der Kommunen auf Selbstverwaltung, wodurch diese ihre Verfahren ebenfalls eigenständig regeln können.
Für bietende Unternehmen hat dies zur Folge, dass sie sich trotz harmonischeren EU-Rechts je nach ausschreibender Kommune und Bundesland immer wieder mit unterschiedlichen Abläufen, technischen Plattformen und Bearbeitungsroutinen auseinandersetzen müssen~\parencite{aufgaben_foederalstaat}.

\subsubsection{Frankreich: Zentralistische Tradition}

Deutschland als Bundesstaat und Frankreich als Einheitsstaat repräsentieren zwei gegensätzliche administrative Modelle.
Während Deutschland durch seinen ausgeprägten Föderalismus geprägt ist, zeichnet sich Frankreich durch eine historisch gewachsene, starke Zentralisierung aus.
Dieser strukturelle Unterschied wirkt sich direkt auf die Umsetzung von EU-Vergabeverfahren aus.
Zwar unterliegen beide Länder denselben EU-Richtlinien, doch profitiert Frankreich von einer wesentlich höheren Standardisierung der Abläufe.
Mit dem Code de la commande publique existiert ein einheitliches Gesetzbuch, das die Vergaberegeln landesweit bündelt und für Bieter transparenter macht.

Ein entscheidender Vorteil gegenüber der deutschen Fragmentierung ist zudem die Existenz starker zentraler Beschaffungsstellen, insbesondere der UGAP (Union des groupements d’achats publics).
Diese Institution bündelt Beschaffungsvolumina und schließt komplexe Rahmenverträge zentral ab.
Für Unternehmen bedeutet dies, dass sie es bei großen Ausschreibungen häufig mit hochprofessionellen, zentralen Ansprechpartnern zu tun haben, statt mit einer Vielzahl kleiner, lokaler Vergabestellen.

Diese administrative Professionalisierung korrespondiert mit einer höheren Akzeptanz digitaler Prozesse.
Laut dem Index für digitale Wirtschaft und Gesellschaft (DESI) 2022 der Europäischen Kommission nutzen rund 87\% der französischen Internetnutzer elektronische Behördendienste, während dieser Wert in Deutschland bei lediglich 55\% liegt.
Die Kombination aus zentraler Steuerung und höherer digitaler Affinität lässt für Frankreich niedrigere Transaktionskosten erwarten als im föderalen Deutschland~\parencite{GovTech_FR}.

\subsubsection{Estland: Digitale Vorreiterschaft}

Mit lediglich 1,3 Millionen Einwohnern (Stand: 2024) ist Estland deutlich kleiner als Frankreich und Deutschland.
Dennoch betreibt Estland das wohl umfassendste digitale Regierungssystem der Welt~\parencite{estonia_powerhouse}.
Es ist das erste Land, welches 100\% ihrer Regierungsdienstleistungen digitalisiert hat.
So können die Bürger ein Unternehmen in unter 15 Minuten anmelden, wählen und sich scheiden lassen, ohne das Haus zu verlassen~\parencite{estonia_powerhouse}.
Diese Entwicklung began bereits im Jahr 1999, als sich Estland dazu entschieden hat nicht die bestehenden Systeme zu nutzen sondern eigene Systeme von Grund auf neu aufzubauen~\parencite{estonia_digital}.

Zwei wesentliche Säulen tragen dabei eine wesentliche Rolle.
\begin{itemize}
\item\textbf{Die nationale digitale ID (e-ID):} Dadurch haben Bürger die Möglichkeit sich digital zu verifizieren und online zu unterschreiben.

\item\textbf{Die X-Road:} Eine interoperable Plattform zum Datenaustausch.
Dadurch können getrennte Datenbanken sicher und standardisiert miteinander kommunizieren.
\end{itemize}

Dazu kommt das \("\)Once-Only\("\)-Prinzip.
Dieses ist gesetzlich verankert und verpflichtet alle Behörden die X-Road zu nutzen.
Dadurch haben die Bürger eine zentrale Stelle, an welcher alle Informationen gesammelt und aktualisiert werden.
Das Vergabesystem kommuniziert somit direkt mit anderen staatlichen Registern\textsuperscript{}.
Für Bieter bedeutet dies, dass die formale Komplexität einer EU-Ausschreibung durch Automatisierung maskiert wird.~\parencite{estonia_powerhouse}.

Diese Maßnahmen haben einen großen Einfluss, nicht nur auf die Bürger, sondern auch auf die Wirtschaft.
Schätzungsweise spart das System der Gesellschaft bis zu 1.400 Arbeitsstunden jährlich und hat einen Einfluss von 4-7\% des jährlichen Bruttoinlandproduktes~\parencite{estonia_powerhouse}.

Dieses hohe Maß an institutionalisierter Effizienz und der Digitalisierung positioniert Estland als einen "Best-Practice"-Fall im europäischen E-Government, dessen administrative Logik sich fundamental von der Deutschlands und Frankreichs unterscheidet.

\subsubsection{Zusammenfassender Vergleich und Implikationen für den Wettbewerb}

Die Analyse der drei Länderkontexte offenbart fundamental unterschiedliche administrative Grundstrukturen, die sich direkt auf die wahrgenommenen Transaktionskosten und die Effizienz des Vergabewesens auswirken.
Die geplante Verfahrensdauer fungiert als Signal an die Unternehmen.
Wie dieses Signal interpretiert wird – ob als Indikator für sachliche Komplexität oder für administrative Ineffizienz – hängt vom Vertrauen in das jeweilige Verwaltungssystem ab.

In Deutschland treffen Unternehmen auf ein hochgradig föderales und fragmentiertes System.
Die Existenz von rund 30.000 Vergabestellen führt dazu, dass selbst bei harmonisiertem EU-Recht die administrative Umsetzung extrem heterogen ist.
Für viele kleinere Stellen sind komplexe EU-Verfahren keine Routine, was zu Unsicherheiten und ineffizienten Abläufen führt.
Eine lange Verfahrensdauer dürfte hier primär als Signal für hohe, unkalkulierbare Bürokratie und mangelnde Professionalisierung interpretiert werden.

In Frankreich agieren Unternehmen in einem zentralistischen System, das durch den Code de la commande publique und starke zentrale Beschaffungsstellen (UGAP) geprägt ist.
Diese Professionalisierung sorgt für eine höhere Standardisierung der Abläufe.
Gepaart mit einer signifikant höheren Akzeptanz digitaler Behördendienste (87 \% Nutzung), erscheint das System für Bieter berechenbarer.
Eine lange Dauer wird hier eher als notwendige Prüfzeit einer professionellen Bürokratie verstanden denn als Ineffizienz.

In Estland trifft die Verfahrensdauer auf ein „Digital-by-Default“-System, das auf maximale Effizienz ausgelegt ist.
Durch das gesetzlich verankerte \enquote{Once-Only}-Prinzip und die X-Road-Infrastruktur werden die Transaktionskosten für Bieter systemisch minimiert, da Nachweise automatisch abgerufen werden.
Eine lange Verfahrensdauer ist in diesem hocheffizienten Umfeld eine Anomalie und signalisiert am ehesten eine hohe inhaltliche Komplexität des Auftrags.

Die in Tabelle 1 zusammengefassten strukturellen Unterschiede stützen die Annahme, dass die abschreckende Wirkung langer Verfahrensdauern kein universelles Phänomen ist, sondern vom nationalen Verwaltungskontext moderiert wird.
Aus diesen theoretischen Vorüberlegungen leitet sich die dritte Hypothese ab.



\begin{table}[H]
\centering
\caption{Administrativ-struktureller Ländervergleich}
\label{tab:admin_vergleich}

\renewcommand{\arraystretch}{1.5} % Erhöht den Zeilenabstand für bessere Lesbarkeit

% --- SPALTENDEFINITION ---
\begin{tabularx}{\textwidth}{@{} >{\raggedright\arraybackslash}p{3.5cm} >{\raggedright\arraybackslash}X >{\raggedright\arraybackslash}X >{\raggedright\arraybackslash}X @{}}
\toprule
\textbf{Merkmal} & \textbf{Deutschland} & \textbf{Frankreich} & \textbf{Estland} \\
\midrule

Administrative Grundstruktur &
Föderal \& dezentral \tabcite{aufgaben_foederalstaat} &
Zentralistischer Einheitsstaat \tabcite{unterschied_FR_DE} &
\enquote{Digital-by-Default}-Staat \tabcite{estonia_powerhouse} \\

Umsetzung \& Prozess-Standardisierung &
Dezentral \& fragmentiert (Mangelnde Routine bei EU-Verfahren) \tabcite{aufgaben_foederalstaat} &
Zentralisiert \& professionalisiert (Starke Bündelung) \tabcite{GovTech_FR} &
Vollständig automatisiert \& datengetrieben \tabcite{estonia_digital} \\

Bündelung \& Effizienz-Prinzipien &
Fragmentiert (ca. 30.000 Stellen) \tabcite{buerokratie_DE} &
Zentral gebündelt (UGAP) \tabcite{GovTech_FR} &
Systemisch (Gesetzl. \enquote{Once-Only}-Prinzip) \tabcite{estonia_powerhouse} \\

Digitale Nutzung (Bürger/ Unternehmen) &
Gering (55\% Nutzer) \tabcite{GovTech_FR} &
Hoch (87\% Nutzer) \tabcite{GovTech_FR} &
Vollständig (99\% e-ID) \tabcite{estonia_powerhouse} \\
\bottomrule
\end{tabularx}

% --- QUELLE / ANMERKUNG (APA-Stil: Linksbündig unter der Tabelle) ---
\vspace{1mm} % Kleiner Abstand
\begin{minipage}{\textwidth} % Hilft, den Text genau so breit wie die Tabelle zu halten
    \footnotesize \textit{Anmerkung.} Eigene Darstellung basierend auf den zitierten Quellen.
\end{minipage}

\end{table}
\clearpage