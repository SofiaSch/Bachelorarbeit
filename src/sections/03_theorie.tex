\section{Theoretischer Hintergrund}\label{sec:theoretischer-hintergrund}

Die Beantwortung der zentralen Forschungsfrage, welche Auswirkungen die Verfahrensdauer auf die Wettbewerbsintensität und die Beteiligung von KMUs hat, bildet den Gegenstand dieses Kapitels.
Zu diesem Zweck wird zunächst der theoretische Hintergrund erörtert.
In der vorliegenden Arbeit werden nach einer Darlegung der allgemeinen Informationen über das Vergabeverfahren die spezifischen Aspekte des Themas beleuchtet, woraus sich schließlich die zu prüfenden Hypothesen ergeben.

\subsection{Die Grundlagen der öffentlichen Auftragsvergabe}\label{subsec:die-grundlagen-der-offentlichen-auftragsvergabe}

Die öffentliche Auftragsvergabe stellt einen der bedeutendsten Wirtschaftsfaktoren moderner Volkswirtschaften dar.
Mit einem Anteil von etwa 12\% am Bruttoinlandsprodukt (BIP) und fast einem Drittel der gesamten Staatsausgaben in den OECD-Ländern werden immense Summen in Infrastruktur, Güter und Dienstleistungen investiert~\parencite{Value_Creation_in_PP}.
Allein in Deutschland flossen im Jahr 2019 rund 35\% der Staatsausgaben über diesen Kanal in die Wirtschaft, wobei Länder und Kommunen die größten Auftraggeber sind~\parencite{Oeffentliche_Auftraege_DE}.

Doch was genau verbirgt sich hinter dem Begriff des öffentlichen Vergabeverfahrens?
Im Allgemeinen handelt es sich um einen Vertrag zwischen der öffentlichen Hand und Unternehmen über die Erbringung von Liefer-, Dienst- oder Bauleistungen.
Als öffentliche Auftragsgeber werden alle Dienstellen des Bundes, der Länder, Gemeindeverbände und sonstige juristische Personen des öffentlichen Rechts, wie beispielsweise Hochschulen, definiert.
Außerdem Einrichtungen, die vom Staat finanziert werden (z.B.\ Krankenhäuser) und weitere Besonderheiten~\parencite{Oeffentliche_Auftraege_DE}.

Im Jahr 2014 wurde ein großer Schritt bezüglich des europäischen Vergaberechts gemacht, da dort neue EU-Vergaberichtlinien verabschiedet wurden~\parencite{Oeffentliche_Auftraege_DE}.
Diese zielen auf eine Vereinfachung der Verfahren und Stärkung der KMU-Beteiligung ab~\parencite{EU2014RL24}.
Dadurch haben sich grundlegende Prinzipien für das Vergabeverfahren etabliert, die das Ziel verfolgen das objektiv beste Angebot mit einem optimalen Preis-Leistung-Verhältnis auszuwählen.
Besonders relevant, auch im Bezug auf die KMU, gelten die 3 Prinzipien: Gleichheit, Nichtdiskriminierung und Transparenz~\parencite{JIM_Fair_Transparent_Competitive}.
Sie sollen garantieren, dass alle Unternehmen die gleiche Chance haben und niemand bevorzugt wird.
Als zentrales Prinzip gilt der Grundsatz des Wettbewerbs, welcher dazu führt, dass alle anderen Prinzipien erreicht werden~\parencite{Oeffentliche_Auftraege_DE}.
Um zudem das ökonomische Ziel zu erfüllen, gilt das Prinzip der Effizienz beziehungsweise Wirtschaftlichkeit~\parencite{JIM_Fair_Transparent_Competitive}.
Dieses Prinzip gewährleistet, dass hinsichtlich der eingesetzten Steuergelder das wirtschaftlichste und finanziell beste Resultat erzielt wird.

Diese Prinzipien bilden die normative Grundlage, auf der das gesamte System der öffentlichen Auftragsvergabe in der EU beruht.

Ein weiteres wichtiges Instrument im Vergaberecht sind die Schwellenwerte.
Diese werden von der EU-Kommission festgelegt und definieren, ob ein Auftrag national oder europaweit ausgeschrieben werden muss.
Bei sehr geringen Auftragswerten ist keine Ausschreibung notwendig, um den bürokratischen Aufwand gering zu halten.
Die Höhe der Schwellenwerte unterscheidet sich je nach Art des Auftrags~\parencite{schwellenwerte}.

In der Auftragsvergabe gibt es mehrere Verfahrensarten.
Der Unterschied besteht im Wesentlichen in ihrer Offenheit und dem Grad des Wettbewerbs.
In der öffentlichen Ausschreibung (beziehungsweise dem offenen Verfahren) werden eine unbegrenzte Anzahl an Unternehmen bekannt gegeben.
Dadurch wird ein uneingeschränkter Wettbewerb garantiert und somit das wirtschaftlich wertvollste Angebot ermittelt.
Alternativ gibt es die beschränkte Ausschreibung (beziehungsweise das nicht offene Verfahren).
Die Anzahl der Unternehmen ist beschränkt, welche von dem Auftraggeber angesprochen werden.
Zusätzlich gibt es noch die Verhandlungsvergabe, wodurch die Auftraggeber den meisten Freiraum haben.
Hier wird direkt mit ausgewählten Unternehmen verhandelt, sie ist jedoch nur unter bestimmten Voraussetzungen zulässig.
Unabhängig von der gewählten Art folgt der Verfahrensablauf einem strukturierten, mehrstufigen Prozess~\parencite{Oeffentliche_Auftraege_DE}.

Der grobe Verfahrensablauf ist in mehrere Schritte unterteilt, wie Abbildung~\ref{fig:verfahrensablauf} veranschaulicht.
Im ersten Schritt erfolgt die Auftragsbekanntmachung.
Je nachdem, ob es sich um ein nationales oder ein europaweites Verfahren handelt, können die Unterlagen entweder direkt heruntergeladen oder angefordert werden.
Anschließend können die Teilnahmeanträge eingereicht und Angebote abgegeben werden.
Die Angebote werden geprüft und bewertet, anschließend wird die Zuschlagserteilung bekannt gegeben.

\begin{figure}[h!]
    \centering
    \includegraphics[width=0.8\textwidth]{images/verfahrensablauf}
    \caption{Der grobe Ablauf des öffentlichen Vergabeverfahrens.}
    \label{fig:verfahrensablauf}
    \footnotesize
    \parencite{Oeffentliche_Auftraege_DE}
\end{figure}

\subsection{Bürokratie als Transaktionskostentreiber}\label{subsec:burokratie-als-transaktionskostentreiber-in-der-offentlichen-vergabe}

Damit die Grundprinzipien wie Transparenz, Wettbewerb und Gleichbehandlung eingehalten werden können erfordert das formalisierte und standardisierte Prozesse.
Diese erfolgen in Regeln, Dokumentationspflichten und mehrstufigen Abläufen und bilden somit die Bürokratie des Vergabewesens.
Was zweifelsfrei notwendig ist, kann dennoch eine Hürde sein, da eine hohe Bürokratie mit hohen Kosten verbunden ist~\parencite{Discretion_supplier_selection}.
In diesem Fall werden die Kosten nicht zwangsläufig in monetären Werten berechnet, sondern in Form von Transaktionskosten.
Das sind alle Kosten, die beim Abschluss eines Geschäfts entstehen~\parencite{transaktionskosten}.
Darunter fallen die direkten Kosten, die sofort sichtbar und messbar sind.
Wie auch die indirekten Kosten, wie zum Beispiel die Such- und Informationskosten oder Verhandlungs- und Entscheidungskosten~\parencite{transaktionskosten_2}.
Zusammengefasst also alle Aufwände, die ein Unternehmen hat, um überhaupt an der Ausschreibung teilnehmen zu können.

In einer Ausschreibung fallen direkte Kosten in Form von Ressourcenbindung an.
Dazu gehört die Kapitalbindung, was insbesondere die Opportunitätskosten erhöht, da dieses Kapital nicht für andere Projekte genutzt werden kann~\parencite{opportunitaetskosten}.
Außerdem die Personalbindung, da sich wichtige Mitarbeiter mit der Ausschreibung befassen und ihre Kapazitäten in das ausarbeiten des Angebots stecken müssen.

Indirekte Kosten treten durch die Signalwirkung auf.
Eine lange Frist gilt oft als Signal für Komplexität, wodurch der Prozess als ineffizient und schwerfällig interpretiert wird.
Zudem kann eine lange Frist für Unsicherheit, sowohl auf der Angebots- als auch auf der Nachfrageseite - je länger der Prozess, desto größer das Risiko, dass sich Marktbedingungen, Kosten oder Anforderungen ändern~\parencite{PPP_transparency}.

Bürokratie verursacht Transaktionskosten, woraus sich die logische Konsequenz ergibt, dass bei steigenden Kosten und Risiken der erwartete Nettonutten für ein Unternehmen sinkt.
Dementsprechend werden sich Unternehmen gegen eine Angebotsabgabe von übermäßig langen Verfahren entscheiden~\parencite{Discretion_supplier_selection}.

Während diese steigenden Transaktionskosten alle Unternehmen betreffen, sind kleine und mittlere Unternehmen aufgrund ihrer spezifischen strukturellen Merkmale davon in besonderem Maße betroffen~\parencite{PPP_transparency}.

\subsection{Die Sondersituationen von kleinen und mittleren Unternehmen (KMU)}\label{subsec:die-sondersituationen-von-kleinen-und-mittleren-unternehmen-(kmu)}

Im Jahr 2011 zählten alleine 99\% aller Unternehmen in Deutschland zu der Gruppe der kleinen und mittleren Unternehmen.
Das umfasst ein Beschäftigungsvolumen von etwa 60\% und 34\% der erzielten Umsätze~\parencite{KMU_relevance}.
Die Zahlen alleine zeigen bereits, dass KMU ein großer und wichtiger Teil der Wirtschaft sind.
Und auch die Erleichterungen durch die Reform zeigen, dass der Staat die relevanz erkannt hat:

\begin{blockquote}
    Die öffentliche Vergabe sollte an die Bedürfnisse von KMU angepasst werden.~\parencite[Erwägungsgrund 78]{EU2014RL24}
\end{blockquote}

So sieht die neue Reform vor, dass große Aufträge in kleinere Teil- und Fachlose aufgeteilt wird, sodass diese auch von kleinen Unternehmen erfüllt werden können~\parencite[Artikel 46]{EU2014RL24}.
Eine andere Maßnahme ist, dass das Unternehmen keine unverhältnismäßig großen Jahresumsatz vorweisen muss, um an der Ausschreibung teilzunehmen.
Seit der Reform darf in der Regel nur noch das Zweifache des Auftragswertes gefordert werden~\parencite[Artikel 58, Absatz 3]{EU2014RL24}.

Obwohl die Richtlinien das Vergabeverfahren für KMU geöffnet hat, besteht weiterhin Hürden, die die Teilnahme erschwert.
Kapital und Personal ist begrenzt und besonders junge und sehr kleine Unternehmen verfügen über wenig Ressourcen, die sie für komplexe Ausschreibungen zur Verfügung haben.
Auch zu externem Kapital ist der Zugang meist begrenzt~\parencite{SME_Instruments}.
Das kann bedeuten, dass KMU weder die Zeit noch das Kapital haben um neben dem Tagesgeschäft weitere Ressourcen für Ausschreibungen zu entbehren.

Ein weiterer Punkt ist die geringere Erfahrung im Umgang mit übermäßig bürokratischen Prozessen~\parencite{JIM_Fair_Transparent_Competitive}.
Das führt zu erhöhter Unsicherheit sich überhaupt erst zu bewerben oder durch formale Fehler ausgeschlossen zu werden.

Diese Kombination von geringeren Ressourcen und erhöhten Ressourcen kann zur folge haben, dass bei einem langen Verfahren mit hohen Transaktionskosten der Abschreckungseffekt überproportional hoch ist.
Der Effekt tritt bei KMU stärker auf, trotz der neuen Reform, was dazu führt, das diese aus dem Wettbewerb verdrängt werden~\parencite{PPP_transparency}.



\subsection{Der Länderspezifische Verwaltungskontext als Moderator}\label{subsec:der-landerspezifische-verwaltungskontext-als-moderator}

% Obwohl Dezentralisierung im Sinne des New Public Managements durch die Schaffung kleiner, strategischer Einheiten die Effizienz steigern kann (Morais & Santos, 2024)\textsuperscript{}, führt die spezifische föderale Struktur Deutschlands in der Praxis zu einer Fragmentierung der Vergabelandschaft.
% Die Vielzahl an Vergabebehörden auf Bundes-, Landes- und Kommunalebene schafft eine heterogene und komplexe Regelungsdichte\textsuperscript{}.
% Für Unternehmen erhöht dies die Transaktionskosten und die Unsicherheit, weshalb anzunehmen ist, dass die abschreckende Wirkung einer langen Verfahrensdauer in Deutschland besonders stark ausgeprägt ist.