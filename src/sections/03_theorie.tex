\section{Theoretischer Hintergrund}\label{sec:theoretischer-hintergrund}

Die Beantwortung der zentralen Forschungsfrage, welche Auswirkungen die Verfahrensdauer auf die Wettbewerbsintensität und die Beteiligung von KMUs hat, bildet den Gegenstand dieses Kapitels.
Zu diesem Zweck wird zunächst der theoretische Hintergrund erörtert.
In der vorliegenden Arbeit werden nach einer Darlegung der allgemeinen Informationen über das Vergabeverfahren die spezifischen Aspekte des Themas beleuchtet, woraus sich schließlich die zu prüfenden Hypothesen ergeben.

\subsection{Die Grundlagen der öffentlichen Auftragsvergabe}\label{subsec:die-grundlagen-der-offentlichen-auftragsvergabe}

Die öffentliche Auftragsvergabe stellt einen der bedeutendsten Wirtschaftsfaktoren moderner Volkswirtschaften dar.
Mit einem Anteil von etwa 12\% am Bruttoinlandsprodukt (BIP) und fast einem Drittel der gesamten Staatsausgaben in den OECD-Ländern werden immense Summen in Infrastruktur, Güter und Dienstleistungen investiert~\parencite{Value_Creation_in_PP}.
Allein in Deutschland flossen im Jahr 2019 rund 35\% der Staatsausgaben über diesen Kanal in die Wirtschaft, wobei Länder und Kommunen die größten Auftraggeber sind~\parencite{Oeffentliche_Auftraege_DE}.

Doch was genau verbirgt sich hinter dem Begriff des öffentlichen Vergabeverfahrens?
Im Allgemeinen handelt es sich um einen Vertrag zwischen der öffentlichen Hand und Unternehmen über die Erbringung von Liefer-, Dienst- oder Bauleistungen.
Als öffentliche Auftragsgeber werden alle Dienstellen des Bundes, der Länder, Gemeindeverbände und sonstige juristische Personen des öffentlichen Rechts, wie beispielsweise Hochschulen, definiert.
Außerdem Einrichtungen, die vom Staat finanziert werden (z.B.\ Krankenhäuser) und weitere Besonderheiten~\parencite{Oeffentliche_Auftraege_DE}.

Der Grund, warum es das Vergabeverfahren gibt basiert auf verschiedenen Prinzipien, mit dem Ziel das objektiv beste Angebot mit einem optimalen Preis-Leistung-Verhältnis auszuwählen.
Zuerst gibt es die 3 Prinzipien: Gleichheit, Nichtdiskriminierung und Transparenz~\parencite{JIM_Fair_Transparent_Competitive}.
Sie sollen garantieren, dass alle Unternehmen die gleiche Chance haben und niemand bevorzugt wird.
Als zentrales Prinzip gilt der Grundsatz des Wettbewerbs, welcher dazu führt, dass alle anderen Prinzipien erreicht werden~\parencite{Oeffentliche_Auftraege_DE}.
Um zudem das ökonomische Ziel zu erfüllen, gilt das Prinzip der Effizienz beziehungsweise Wirtschaftlichkeit~\parencite{JIM_Fair_Transparent_Competitive}.
Dieses Prinzip gewährleistet, dass hinsichtlich der eingesetzten Steuergelder das wirtschaftlichste und finanziell beste Resultat erzielt wird.

Diese Prinzipien bilden die normative Grundlage, auf der das gesamte System der öffentlichen Auftragsvergabe in der EU beruht.

Ein weiteres wichtiges Instrument im Vergaberecht sind die Schwellenwerte.
Diese werden von der EU-Kommission festgelegt und definieren, ob ein Auftrag national oder europaweit ausgeschrieben werden muss.
Bei sehr geringen Auftragswerten ist keine Ausschreibung notwendig, um den bürokratischen Aufwand gering zu halten.
Die Höhe der Schwellenwerte unterscheidet sich je nach Art des Auftrags~\parencite{schwellenwerte}.

Laut Vergaberecht gibt es mehrere Verfahrensarten, welche je nach dem, ob der Wert unter oder oberhalb

Der grobe Verfahrensablauf

% Eu Rechtsrahmen -> Eu Richtlinien (Effizienz, Wettbewerb, Transparenz, Gleichbehandlung, Nicht-Diskriminierung)

\subsection{Bürokratie als Transaktionskostentreiber in der öffentlichen Vergabe}\label{subsec:burokratie-als-transaktionskostentreiber-in-der-offentlichen-vergabe}

\subsection{Die Sondersituationen von kleinen und mittleren Unternehmen (KMU)}\label{subsec:die-sondersituationen-von-kleinen-und-mittleren-unternehmen-(kmu)}


\subsection{Der Länderspezifische Verwaltungskontext als Moderator}\label{subsec:der-landerspezifische-verwaltungskontext-als-moderator}