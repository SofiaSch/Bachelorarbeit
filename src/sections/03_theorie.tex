\section{Theoretischer Hintergrund und Hypothesenentwicklung}
% Hier stellen Sie die zentralen Theorien vor (z.B. Transaktionskostentheorie)[cite: 2440].
% Wichtige Begriffe definieren, falls nicht schon in der Einleitung geschehen[cite: 2442].
% Jede Hypothese muss sorgfältig und nachvollziehbar aus der Theorie und bisherigen Forschung abgeleitet werden[cite: 2444, 2445].

\subsection{Der Einfluss der Verfahrensdauer auf die Wettbewerbsintensität}
% Argumentation für H1 ...

\textit{Hypothese 1: Je länger die geplante Verfahrensdauer einer öffentlichen Ausschreibung ist, desto geringer ist die Anzahl der abgegebenen Gebote.}

\subsection{Der Einfluss der Verfahrensdauer auf die KMU-Beteiligung}
% Argumentation für H2 ...

\textit{Hypothese 2: Je länger die geplante Verfahrensdauer ist, desto geringer ist der Anteil der Gebote von kleinen und mittleren Unternehmen (KMU).}

\subsection{Der moderierende Einfluss des Länderkontexts}
% Argumentation für H3 ...

\textit{Hypothese 3: Der negative Einfluss der geplanten Verfahrensdauer auf den Wettbewerb ist in Deutschland am stärksten, in Frankreich moderat und in Estland am schwächsten ausgeprägt.}

% Hier können Sie ein grafisches Modell Ihres Forschungsdesigns einfügen[cite: 2447].
% Siehe \ref{fig:modell} in main.tex