\section{Diskussion}
% Hier "bewerben" Sie Ihre Ergebnisse und zeigen, warum sie wichtig sind[cite: 2465, 2466].
% Die Diskussion folgt einer Pyramidenform (vom Spezifischen zum Allgemeinen)[cite: 2487].

\subsection{Zusammenfassung und Interpretation der Ergebnisse}
% Kurze Zusammenfassung der Hauptergebnisse in einem Absatz[cite: 2467].

\subsection{Theoretische Implikationen}
% Was bedeuten Ihre Ergebnisse für die bestehende Theorie (z.B. TKT)?[cite: 2469].
% Greifen Sie die in der Einleitung formulierten Beiträge auf[cite: 2470].

\subsection{Praktische Implikationen}
% Welche konkreten Handlungsempfehlungen ergeben sich für wen (z.B. Vergabestellen, Politik)?[cite: 2472].
% Mit Beispielen untermauern[cite: 2473].

\subsection{Limitationen und zukünftige Forschung}
% Seien Sie ehrlich über die Schwächen Ihrer Arbeit (z.B. Datenqualität, Kausalität)[cite: 2474].
% Welche Forschungsfragen ergeben sich aus Ihrer Arbeit?[cite: 2477].