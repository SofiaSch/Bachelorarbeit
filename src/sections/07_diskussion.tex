\newpage
\section{Diskussion}\label{sec:diskussion}

\subsection{Zusammenfassung und Interpretation der Ergebnisse}\label{subsec:zusammenfassung-und-interpretation-der-ergebnisse}
Das Ziel dieser Arbeit ist die Beantwortung der Forschungsfrage, inwiefern die geplante Dauer des öffentlichen Vergabeverfahrens die Wettbewerbsintensität beeinflusst.
Diese wird anhand der Gesamtzahl der Gebote sowie der spezifischen Beteiligung von Klein- und mittelständischen Unternehmen in Deutschland, Estland und Frankreich gemessen.
Die Ergebnisse der Regressionsanalyse liefern ein differenziertes Bild, das die universelle Anwendbarkeit der Transaktionskostentheorie in diesem Bereich infrage stellt.

Für die erste Hypothese, welche sich auf die Wettbewerbsintensität fokussiert, konnte ein negativer Zusammenhang zwischen Verfahrensdauer und Gebotsanzahl für die Gesamtheit der Länder nicht bestätigt werden.
Während die Theorie davon ausgeht, dass längere Fristen höhere Opportunitätskosten und Unsicherheiten signalisieren und damit abschreckend wirken, zeigen die Daten, dass in Deutschland und Frankreich tendenziell positive Effekte bestehen.
In Estland besteht kein nachweisbarer Zusammenhang zwischen der Dauer und der Anzahl der Gebote.
Dies deutet darauf hin, dass Unternehmen in großen Volkswirtschaften längere Fristen eher als notwendigen Zeitraum für die Erstellung qualitativ hochwertiger und komplexer Angebote wahrnehmen, statt als bürokratische Hürde.

Die Ergebnisse zur KMU-Beteiligung, die in der zweiten Hypothese erforscht wurde, sind besonders in Estland aufschlussreich.
Hier bestätigt sich ein überproportionaler Abschreckungseffekt ausschließlich im estnischen Kontext.
Dort sinkt der Anteil der KMU-Gebote mit zunehmender Verfahrensdauer signifikant.
Dies lässt sich durch die hohe digitale Effizienz des estnischen Systems erklären.
Mit einer Struktur, die auf maximale Geschwindigkeit und dem \enquote{Once-Only}- Prinzip ausgelegt ist, werden Zeitverzögerungen intensiver als Ressourcenbelastung interpretiert, wovon KMU aufgrund ihrer begrenzten Mittel stärker betroffen sind als Großunternehmen.
In Deutschland und Frankreich lässt sich dieser Effekt nicht beobachten.

Hinsichtlich des Ländervergleichs in der dritten Hypothese muss die ursprüngliche Annahme, dass Deutschland aufgrund seiner föderalen Komplexität am stärksten negativ auf die Verfahrensdauer reagiert, abgelehnt werden.
Entgegen der Hypothese reagiert das hocheffiziente System Estland am sensibelsten auf die zeitlichen Ausdehnungen.
Während in Deutschland und Frankreich die Interaktionseffekte den negativen Basiseffekt neutralisieren, wirkt die Verfahrensdauer in Estland als die stärkste Barriere für den Wettbewerb der KMU.
Die Ergebnisse unterstreichen somit, dass die Wahrnehmung von Bürokratie und Transaktionskosten untrennbar mit dem nationalen Verwaltungskontext und dem damit verbundenen Vertrauen in das System verknüpft ist.

\begin{table}[htbp]
  \centering
  \begin{threeparttable}
  \small
  \caption{Zusammenfassende Bewertung der Hypothesentests}
  \label{tab:hypothesen_bewertung}
  \begin{tabularx}{\textwidth}{>{\raggedright\arraybackslash}p{4cm}XXXX}
    \toprule
    \textbf{Hypothese} & \textbf{Estland} & \textbf{Frankreich} & \textbf{Deutschland} & \textbf{Gesamt}\\
    \midrule
    \textbf{H1:} Längere Dauer $\rightarrow$ Weniger Gebote & Nicht bestätigt (negativ, n.s.) & Abgelehnt (positiver Effekt) & Abgelehnt (positiver Trend) & \textbf{Abgelehnt} \\
    \addlinespace
    \textbf{H2:} Längere Dauer $\rightarrow$ Geringerer KMU-Anteil & Bestätigt (signifikant negativ) & Nicht bestätigt (nahe Null) & Nicht bestätigt (nahe Null) & \textbf{Abgelehnt} (nur länderspezifischer Effekt in EE) \\
    \addlinespace
    \textbf{H3:} Effekt in DE am stärksten, in EE am schwächsten & Abgelehnt (EE reagiert am sensibelsten) & Abgelehnt & Abgelehnt & \textbf{Abgelehnt} \\
    \bottomrule
  \end{tabularx}
  \begin{tablenotes}
    \footnotesize
    \item \textit{Anmerkung:} n.s. = nicht signifikant. Die Bewertung basiert auf den Ergebnissen der ZINB- und Fractional-Logit-Regressionsmodelle.
  \end{tablenotes}
  \end{threeparttable}\label{tab:table}
\end{table}

\subsection{Theoretische Implikationen}\label{subsec:theoretische-implikationen}

Für die Literatur zur öffentlichen Auftragsvergabe leistet diese Arbeit einen wesentlichen Beitrag, indem sie die Grenzen rein ökonomischer Erklärungsmodelle aufzeigt und die Bedeutung des institutionellen Kontextes hervorhebt.

Die klassische Transaktionskostentheorie zeigt, dass administrative Hürden und zeitliche Verzögerungen die Transaktionskosten erhöhen und damit die Attraktivität eines Marktes mindern~\parencite{Discretion_supplier_selection}.
In diesem Sinne wurde für H1 und H2 ein negativer Effekt der Verfahrensdauer auf den Wettbewerb erwartet.
Allerdings deutet die Ablehnung dieser Hypothese in Deutschland und Frankreich darauf hin, dass Zeit in komplexen Verwaltungssystemen nicht als reine Kostenbelastung, sondern als notwendige Vorbereitungszeit wahrgenommen wird.
Dies deckt sich mit der Argumentation in \textcite{buerokratie_DE}, wonach gerade für den Mittelstand eine ausreichende Fristsetzung essenziell ist, um die bürokratischen Anforderungen einer Ausschreibung überhaupt bewältigen zu können.
Somit ist Zeit nicht per se ein negativer Faktor, sondern ihre Wirkung ist davon abhängig, ob sie als administrative Ineffizienz oder als Ermöglichungsraum für Qualität interpretiert wird.

Ein wesentlicher theoretischer Beitrag dieser Arbeit besteht in der Verknüpfung der Transaktionskostentheorie und der Signaltheorie.
Die Ergebnisse zeigen, dass die geplante Verfahrensdauer als Signal für die Effizienz der Verwaltung dient.
In Estland wird eine lange Dauer als Signal für administrative Anomalien interpretiert.
Hier wirkt die Zeit abschreckend, was die Theorie unterstützt, wonach in effizienten Systemen Zeitabweichungen das wahrgenommene Risiko erhöhen.
In Deutschland und Frankreich hingegen scheint die Dauer ein schwaches Signal zu sein, da die Marktakteure aufgrund der föderalen Komplexität bzw.\ zentralistischen Struktur bereits eine gewisse Trägheit antizipieren~\parencite{GovTech_FR}.

\subsection{Praktische Implikationen}\label{subsec:praktische-implikationen}

Aus den Ergebnissen lassen sich gezielte Handlungsempfehlungen für Entscheidungsträger in der öffentlichen Verwaltung und der Wirtschaftspolitik ableiten, um den Wettbewerb nachhaltig zu fördern und die KMU-Beteiligung zu stärken.

\begin{enumerate}
  \item Entgegen der verbreiteten Ansicht, jede zeitliche Ausdehnung eines Vergabeverfahrens sei eine rein bürokratische Hürde, zeigen die Ergebnisse für Deutschland und Frankreich, dass längere Fristen den Wettbewerb nicht zwingend schwächen.
        Auftraggeber sollten die Verfahrensdauer daher nicht als rein negativen Kostenfaktor, sondern als strategisches Instrument zur Qualitätssteigerung betrachten.
        Besonders bei komplexen Aufträgen ermöglicht eine großzügigere Fristsetzung KMU erst die notwendige Zeit für eine fundierte Kalkulation und die Bildung von Bietergemeinschaften.
  \item Die Analyse verdeutlicht, dass in hocheffizienten Kontexten wie in Estland jede Verzögerung überproportional abschreckend auf KMU wirkt.
        In digitalen Pioniersystemen müssen administrative Verzögerungen daher minimiert oder proaktiv kommuniziert werden.
        Da KMU in diesen Systemen auf eine hohe Prozessgeschwindigkeit angewiesen sind, signalisieren ungeplante Zeitverzögerungen ein finanzielles Risiko, das sie im Gegensatz zu Großunternehmen oft nicht tragen können.
  \item Da die Wahrnehmung der Verfahrensdauer stark vom Vertrauen in die administrative Infrastruktur abhängt, sollten Verwaltungen, insbesondere im föderal fragmentierten System Deutschlands, an ihrer Signalqualität arbeiten.
        Neben der technischen E-Vergabe ist die Standardisierung von Abläufen über Bundesländergrenzen hinweg essenziell, um Transaktionskosten vorhersehbarer zu machen.
        Eine aktive Kommunikation der Gründe für längere Fristen kann verhindern, dass Zeit fälschlicherweise als Signal für Ineffizienz interpretiert wird.
  \item Mit rund 30.000 Vergabestellen weist Deutschland eine extrem heterogene Landschaft auf, im Gegensatz zur Bündelung in Frankreich (UGAP) oder dem \enquote{Digital-by-Default}-System in Estland.
        Langfristig sollte Deutschland die Einführung zentralerer Vergabestrukturen prüfen.
        Eine stärkere Bündelung würde nicht nur die Professionalisierung steigern, sondern auch das Fachpersonal vor Ort massiv entlasten.
        Durch zentrale Plattformen und automatisierte Standardprozesse könnten repetitive Aufgaben reduziert werden, wodurch sich Fachkräfte auf die qualitativ hochwertige Gestaltung komplexer Ausschreibungen konzentrieren können.
\end{enumerate}

Für die deutsche Vergabepraxis implizieren die Ergebnisse, dass eine reine Erhöhung der Fristen zur KMU-Förderung zu kurz greift.
Die digitale Transformation muss so gestaltet werden, dass sie die prozessualen Transaktionskosten senkt, bis Zeit kein kritischer Faktor mehr für die Beteiligung darstellt.
Erst wenn die administrative Last, analog zum estnischen Vorbild, minimiert ist, können Fristen verkürzt werden, um die Gesamteffizienz des öffentlichen Sektors zu steigern, ohne den Wettbewerb zu gefährden.

\subsection{Limitationen und zukünftige Forschung}\label{subsec:limitationen-und-zukunftige-forschung}

Trotz der signifikanten Befunde weist die vorliegende Arbeit Limitationen auf, die bei der Interpretation zu berücksichtigen sind und zugleich Ansatzpunkte für zukünftige Forschung bieten.

Die Analyse basiert auf Daten von OpenTender.eu.
Obwohl dies eine umfassende Quelle darstellt, besteht eine Abhängigkeit von der Meldequalität nationaler Behörden.
Der Ausschluss von ca.\ 77\% der Datensätze aufgrund fehlender Auftragswerte stellt eine statistische Limitation dar, ist jedoch inhaltlich von hoher Relevanz, da er eine eklatante Transparenzlücke dokumentiert.
Aus Sicht der Transaktionskostentheorie erschwert dies die Marktsondierung für KMU massiv und stellt eine implizite Markteintrittsbarriere dar.
Zukünftige Studien sollten versuchen, diese Lücken durch den direkten Zugriff auf nationale Primärquellen oder Behördenabfragen zu validieren.

Es besteht die Herausforderung, dass die geplante Verfahrensdauer nicht zufällig festgelegt wird, sondern oft mit der Projektkomplexität korreliert.
Trotz der Verwendung von Kontrollvariablen kann nicht vollständig ausgeschlossen werden, dass die Effekte teilweise auf die zugrunde liegende Schwierigkeit der Aufträge zurückzuführen sind.
Zukünftige Forschung könnte hier mittels Quasi-Experimenten (z.B.\ Gesetzesänderungen bei Fristen) arbeiten, um den isolierten Effekt der Zeit präziser zu isolieren.

Während die quantitative Analyse statistische Belege für länderspezifische Unterschiede liefert, erklärt sie die psychologischen Motive der Bieter nur indirekt über die Signaltheorie.
Ob Unternehmen die Dauer tatsächlich als Vorbereitungschance oder als bürokratisches Hindernis interpretieren, bleibt eine theoretische Ableitung.
Hier wäre qualitative Anschlussforschung (z.B.\ Experteninterviews mit KMU-Entscheidern) wertvoll, um die hergeleiteten Signalwirkungen direkt empirisch zu bestätigen.

Schließlich konzentrierte sich diese Arbeit auf die Beteiligungsintensität.
Zukünftige Studien sollten untersuchen, ob die Verfahrensdauer auch die tatsächlichen Zuschlagschancen beeinflusst.
Es bleibt zu prüfen, ob eine längere Dauer der KMU zwar zur Gebotsabgabe ermutigt, sie im finalen Auswahlprozess im Vergleich zu Großunternehmen dennoch benachteiligt werden.