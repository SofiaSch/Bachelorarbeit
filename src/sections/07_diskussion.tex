\section{Diskussion}




\subsection{Zusammenfassung und Interpretation der Ergebnisse}\label{subsec:zusammenfassung-und-interpretation-der-ergebnisse}
Das Ziel dieser Arbeit ist die Beantwortung der Forschungsfrage, inwiefern die geplante Dauer des öffentlichen Vergabeverfahrens die Wettbewerbsintensität beeinflusst.
Diese wird anhand der Gesamtzahl der Gebote sowie der spezifischen Beteiligung von Klein und Mittelständigen Unternehmen in Deutschland, Estland und Frankreich gemessen.
Die Ergebnisse der Regressionsanalyse liefern ein differenziertes Bild, das die universelle Anwendbarkeit der Transaktionskostentheorie in diesem Bereich infrage stellt.

Für die erste Hypothese, welche sich auf die Wettbewerbsintensität fokussiert, konnte ein negativer Zusammenhang zwischen Verfahrensdauer und Gebotsanzahl für die Gesamtheit der Länder nicht bestätigt werden.
Während die Theorie davon ausgeht, dass längere Fristen höhere Opportunitätskosten und Unsicherheiten signalisiert und somit abschreckend wird, zeigen die Daten, dass in Deutschland und Frankreich tendenziell positive Effekte.
In Estland besteht kein nachweisbarer Zusammenhang zwischen Dauer und Anzahl der Gebote.
Dies deutet darauf hin, dass Unternehmen in großen Volkswirtschaften längere Fristen eher als notwendigen Zeitraum für die Erstellung qualitativ hochwertiger und komplexer Angebote wahrnehmen, statt als bürokratische Hürde.

Die Ergebnisse zur KMU-Beteiligung, welche in der zweiten Hypothese erforscht wurde, sind besonders in Estland aufschlussreich.
Hier bestätigt sich ein überproportionaler Abschreckungseffekt ausschließlich im estnischen Kontext.
Dort sinkt der Anteil der KMU-Gebote mit zunehmender Verfahrensdauer signifikant.
Dies lässt sich durch die hohe digitale Effizienz des estnischen Systems erklären.
Mit einer Struktur, die auf maximale Geschwindigkeit und dem \("\)Once-Only\("\)- Prinzip ausgelegt ist, werden Zeitverzögerungen intensiver als Ressourcenbelastung interpretiert, wovon KMU aufgrund ihrer begrenzten Mittel stärker betroffen sind als Großunternehmen.
In Deutschland und Frankreich lässt sich dieser Effekt nicht beobachten.

Hinsichtlich des Ländervergleichs in der dritten Hypothese muss die ursprüngliche Annahme, dass Deutschland aufgrund seiner föderalen Komplexität am stärksten negativ auf die Verfahrensdauer reagiert, abgelehnt werden.
Entgegen der Hypothese reagiert das hocheffiziente System Estland am sensibelsten auf die zeitliche Ausdehnungen.
Während in Deutschland und Frankreich die Interaktionseffekte den negativen Basiseffekt neutralisieren, wirkt die Verfahrensdauer in Estland als stärkste Barriere für den Wettbewerb durch KMU.
Die Ergebnisse unterstreichen somit, dass die Wahrnehmung von Bürokratie und Transaktionskosten untrennbar mit dem nationalen Verwaltungskontext und dem damit verbundenen Vertrauen in das System verknüpft ist.

\begin{table}[htbp]
  \centering
  \begin{threeparttable}
  \small
  \caption{Zusammenfassende Bewertung der Hypothesentests}
  \label{tab:hypothesen_bewertung}
  \begin{tabularx}{\textwidth}{>{\raggedright\arraybackslash}p{4cm}XXXX}
    \toprule
    \textbf{Hypothese} & \textbf{Estland} & \textbf{Frankreich} & \textbf{Deutschland} & \textbf{Gesamt}\\
    \midrule
    \textbf{H1:} Längere Dauer $\rightarrow$ Weniger Gebote & Nicht bestätigt (negativ, n.s.) & Abgelehnt (positiver Effekt) & Abgelehnt (positiver Trend) & \textbf{Abgelehnt} \\
    \addlinespace
    \textbf{H2:} Längere Dauer $\rightarrow$ Geringerer KMU-Anteil & Bestätigt (signifikant negativ) & Nicht bestätigt (nahe Null) & Nicht bestätigt (nahe Null) & \textbf{Teilweise bestätigt (nur EE)} \\
    \addlinespace
    \textbf{H3:} Effekt in DE am stärksten, in EE am schwächsten & Abgelehnt (EE reagiert am sensibelsten) & Abgelehnt & Abgelehnt & \textbf{Abgelehnt} \\
    \bottomrule
  \end{tabularx}
  \begin{tablenotes}
    \footnotesize
    \item \textit{Anmerkung:} n.s. = nicht signifikant. Die Bewertung basiert auf den Ergebnissen der ZINB- und Fractional-Logit-Regressionsmodelle.
  \end{tablenotes}
  \end{threeparttable}\label{tab:table}
\end{table}

\subsection{Theoretische Implikationen}\label{subsec:theoretische-implikationen}
Für die Literatur der öffentlichen Auftragsvergabe bietet diese Arbeit einen wesentlichen Beitrag, indem sie die Grenzen rein ökonomischer Erklärungsmodelle aufzeigen und die Bedeutung des institutionellen Kontextes hervorheben.

Die klassische Transaktionskostentheorie zeigt auf, dass administrative Hürden und zeitliche Verzögerungen die Transaktionskosten erhöhen und somit die Attraktivität eines Marktes mindern~\parencite{Discretion_supplier_selection}.
In diesem Sinne wurde in H1 und H2 ein negativer Effekt der Verfahrensdauer auf den Wettbewerb erwartet.
Allerdings deutet die Ablehnung dieser Hypothese in Deutschland und Frankreich darauf hin, dass Zeit in komplexen Verwaltungssystemen nicht als reine Kostenbelastung, sondern als notwendige Vorbereitungszeit wahrgenommen wird.
Dies deckt sich mit der Argumentation vom \parencite{buerokratie_DE}, dass gerade für den Mittelstand eine ausreichende Fristsetzung essenziell ist, um die bürokratischen Anforderungen einer Ausschreibung überhaupt bewältigen zu können.
Somit ist Zeit nicht per se ein negativer Faktor, sondern ihre Wirkung ist davon abhängig, ob sie als administrative Ineffizienz oder als Ermöglichungsraum für Qualität interpretiert wird.

Ein wesentlicher theoretischer Beitrag dieser Arbeit liegt in der Verknüpfung der Transaktionskostentheorie und der Signaltheorie.
Die Ergebnisse zeigen, dass die geplante Verfahrensdauer als Signal für die Effizienz der Verwaltung fungiert.
In Estland wird eine lange Dauer als Signal für administrative Anomalien interpretiert.
Hier wirkt die Zeit abschreckend, was die Theorie unterstützt, dass in effizienten Systemen Zeitabweichungen das wahrgenommene Risiko erhöhen.
In Deutschland und Frankreich hingegen scheint die Dauer ein schwaches Signal zu sein, da die Marktakteure aufgrund der föderalen Komplexität bzw.\ zentralistischen Struktur bereits eine gewisse Trägheit antizipieren~\parencite{GovTech_FR}.

\subsection{Praktische Implikationen}\label{subsec:praktische-implikationen}

Um den Wettbewerb zu fördern und die Beteiligung von KMU zu stärken lassen sich gezielte Empfehlungen für Entscheidungsträger in der öffentlichen Verwaltung und Wirtschaftspolitik ableiten.

\begin{enumerate}
  \item Entgegen der oft vertretenen Ansicht, dass jede zeitliche Ausdehnung eines Vergabeverfahrens eine rein bürokratische Hürde darstellt, zeigen die Ergebnisse für Deutschland und Frankreich, dass längere Fristen den Wettbewerb nicht zwingend schwächen.
        Daraus schließt sich, dass Auftraggeber die Verfahrensdauer nicht als rein negativen Kostenfaktor betrachten, sondern als strategisches Instrument zur Qualitätssteigerung.
        Besonders bei komplexen Aufträgen ermöglicht eine großzügigere Fristsetzung KMU erst die notwendige Zeit für eine sorgfältige Kalkulation und die Bildung von Bietergemeinschaften.
  \item Die Analyse verdeutlicht, dass in hocheffizienten, digitalen Verwaltungskontexten wie in Estland jede Verzögerung überproportional abschreckend auf KMU wirkt.
        Somit lässt sich daraus ableiten, dass in digitalen Pilotensystemen administrative Verzögerungen minimiert oder klar kommuniziert werden müssen.
        Da KMU in diesen Systemen auf eine hohe Prozessgeschwindigkeit angewiesen sind, signalisieren ungeplante Zeitverzögerungen ein finanzielles Risiko, das sie im Gegensatz zu Großunternehmen oft nicht tragen können.
  \item Proaktive Signalsteuerung und Vertrauensbildung
  \item Zentralisierung und Professionalisierung nach internationalem Vorbild
\end{enumerate}


\subsection{Limitationen und zukünftige Forschung}\label{subsec:limitationen-und-zukunftige-forschung}
% Seien Sie ehrlich über die Schwächen Ihrer Arbeit (z.B. Datenqualität, Kausalität)[cite: 2474].
% Welche Forschungsfragen ergeben sich aus Ihrer Arbeit?[cite: 2477].