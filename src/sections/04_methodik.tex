\section{Methodik}

\subsection{Foschungsdesign}

\subsection{Datensatz und Stichprobe}

\subsubsection{Datenquelle und -gewinnung}

Die Daten, die für die empirische Analyse genutzt werden stammen aus der OpenTender Plattform, welche vom Government Transparency Institute gesammelt, transformiert und veröffentlicht wird~\parencite{GTI2024}.
In der Datenbank befindet sich die Daten von insgesamt 35 Gerichtsbarkeiten, davon 27 Mitgliedstaaten der Europäischen Union~\parencite{GTI2024}.

Die Daten werden unter anderem im JSON Format zur Verfügung gestellt.
Der Untersuchungszeitraum ist auf die Jahre von 2014 bis 2022 festgelegt.
In 2014 wurde ein großer Schritt bezüglich des eurpäischen Vergaberechts gemacht, da dort neue EU-Vergaberichtlinien verabschiedet wurden.
Diese zielen auf eine Vereinfachung der Verfahren und Stärkung der KMU-Beteiligung ab~\parencite{EU2014RL24}.
Die Wahl für das Jahr 2022 beruht auf der Datenverfügbarkeit, da das neusten Daten für Frankreich aus dem Jahre 2022 sind.
Die Analyse dieses Zeitraums ermöglicht die Analyse innerhalb des aktuellen rechtlichen Rahmens auf einer gemeinsamen Grundlage.

\subsubsection{Stichprobenziehung und Länderauswahl}
Deutschland Förderalismus~\parencite{DE_FOER}

Frankreich Zentralistisch~\parencite{FR_Zentr}

Estland Digital~\parencite{EST_Digital}

\subsection{Variablen und deren Operationalisierung}
% Genaue Beschreibung, wie die UV, die AVs und die Kontrollvariablen gemessen wurden.

\subsection{Statistische Auswertungsmethode}
% Welche statistischen Verfahren wurden genutzt? (z.B. multiple Regressionsanalyse).