\newpage
\appendix

% 1. Haupteintrag "A Anhang" ins Inhaltsverzeichnis zwingen
\addcontentsline{toc}{section}{A Anhang}
\section*{A Anhang}
\label{sec:anhang}

% 2. Magie: Jede \caption im Anhang wird automatisch ein TOC-Eintrag (Subsection-Ebene)
\makeatletter
\let\oldcaption\caption
\renewcommand{\caption}[1]{%
    \oldcaption{#1}%
    \addcontentsline{toc}{subsection}{\thetable \hspace{0.5em} #1}%
}
\makeatother

% Nummerierung auf A1, A2... umstellen
\renewcommand{\thetable}{A\arabic{table}}
\setcounter{table}{0}

% --- Tabelle A1: Stichprobenreduktion ---
\begin{table}[htbp]
    \centering
    \begin{threeparttable}
        \caption{Detaillierte Übersicht der Stichprobenreduktion nach Ländern}
        \label{tab:sample_reduction_appendix}
        \begin{tabular*}{\textwidth}{l @{\extracolsep{\fill}} rrr}
            \toprule
            Land & Rohdaten ($N_{initial}$) & Finale Stichprobe ($N_{final}$) & Verlust (\%) \\
            \midrule
            Deutschland & 303.770 & 55.587 & 81,7\% \\
            Frankreich & 342.021 & 91.149 & 73,3\% \\
            Estland & 29.609 & 3.404 & 88,5\% \\
            \midrule
            \textbf{GESAMT} & \textbf{675.400} & \textbf{150.140} & \textbf{77,8\%} \\
            \bottomrule
        \end{tabular*}
        \begin{tablenotes}[flushleft, para]
            \vspace{0.1cm}
            \footnotesize
            \textit{Anmerkung:} Die Reduktion resultiert primär aus fehlenden Angaben zum Auftragswert sowie zur Verfahrensart.
        \end{tablenotes}
    \end{threeparttable}\label{tab:table1}
\end{table}

% --- Tabelle A2: Korrelationsmatrix ---
\begin{table}[htbp]
    \centering
    \begin{threeparttable}
        \caption{Mittelwerte, Standardabweichungen und Korrelationen der Kernvariablen}
        \label{tab:correlation_appendix}
        \begin{tabular*}{\textwidth}{l @{\extracolsep{\fill}} cc cccc}
            \toprule
            Variable & M & SD & 1 & 2 & 3 & 4 \\
            \midrule
            1. Gebotsanzahl (total\_bids) & 0,51 & 2,17 & 1,00 &  &  &  \\
            2. KMU-Anteil (sme\_share) & 0,58 & 0,46 & 0,04 & 1,00 &  &  \\
            3. Verf.-dauer (z\_duration) & -0,02 & 0,86 & 0,01 & -0,02 & 1,00 & \\
            4. Auftragswert (z\_value) & 0,01 & 0,99 & 0,01 & -0,04 & 0,05 & 1,00 \\
            \bottomrule
        \end{tabular*}
        \begin{tablenotes}[flushleft, para]
            \vspace{0.1cm}
            \footnotesize
            \textit{Anmerkung:} $N = 150.140$ (sme\_share basiert auf $n = 19.760$ für Fälle mit Gebotsanzahl $> 0$).
            \tabcite{Value_Creation_in_PP, Oeffentliche_Auftraege_DE}
        \end{tablenotes}
    \end{threeparttable}\label{tab:table2}
\end{table}

% --- Tabelle A3: Robustheitsprüfung 1 ---
\begin{table}[htbp]
    \centering
    \begin{threeparttable}
        \caption{Robustheitsprüfung 1: ZINB-Modell unter Einbeziehung der Zuschlagskriterien}
        \label{tab:robustness_award}
        \small
        \begin{tabular*}{\textwidth}{l @{\extracolsep{\fill}} cccc}
            \toprule
            Variable & Koeffizient & Std. Fehler & z-Wert & P$>|z|$ \\
            \midrule
            \textit{Count Model (Negative Binomial)} & & & & \\
            Intercept (Referenz: Estland) & -2,8796*** & 0,093 & -30,88 & 0,000 \\
            Land: Frankreich & -1,3461*** & 0,060 & -22,39 & 0,000 \\
            Land: Deutschland & 0,2006*** & 0,059 & 3,41 & 0,001 \\
            Verfahrensdauer (z-std.) & 0,0006 & 0,064 & 0,01 & 0,993 \\
            Dauer $\times$ Frankreich & 0,0997 & 0,067 & 1,49 & 0,138 \\
            Dauer $\times$ Deutschland & 0,0669 & 0,065 & 1,02 & 0,304 \\
            Auftragswert (log, z-std.) & -0,0094 & 0,010 & -0,90 & 0,370 \\
            Zuschlagskriterium (Gewichtet) & -0,4195*** & 0,028 & -14,87 & 0,000 \\ \addlinespace
            \textit{Kontrollvariablen \& Zeit-Effekte} & & & & \\
            Verfahrensart (Selektiv) & -0,1109*** & 0,028 & -3,96 & 0,000 \\
            Sektor (Dienstleistung) & 0,3857*** & 0,023 & 16,60 & 0,000 \\
            Sektor (Bauleistung) & 0,7124*** & 0,032 & 22,53 & 0,000 \\
            Jahres-Fixed-Effects (2015-2022) & \multicolumn{4}{c}{Inkludiert (Alle signifikant $p < 0,01$)} \\
            \midrule
            Log-Likelihood & -95.818 & & & \\
            Beobachtungen (N) & 150.140 & & & \\
            \bottomrule
        \end{tabular*}
        \begin{tablenotes}[flushleft, para]
            \vspace{0.1cm}
            \footnotesize
            \textit{Anmerkung:} * $p < 0,05$, ** $p < 0,01$, *** $p < 0,001$.
        \end{tablenotes}
    \end{threeparttable}\label{tab:table3}
\end{table}

% --- Tabelle A4: Robustheitsprüfung 2 ---
\begin{table}[htbp]
    \centering
    \begin{threeparttable}
        \caption{Robustheitsprüfung 2: ZINB-Modell isoliert für den Dienstleistungssektor}
        \label{tab:robustness_services}
        \small
        \begin{tabular*}{\textwidth}{l @{\extracolsep{\fill}} cccc}
            \toprule
            Variable & Koeffizient & Std. Fehler & z-Wert & P$>|z|$ \\
            \midrule
            Intercept (Referenz: Estland) & -2,9593*** & 0,158 & -18,73 & 0,000  \\
            Land: Frankreich & -1,6864*** & 0,096 & -17,57 & 0,000  \\
            Land: Deutschland & 0,0226 & 0,098 & 0,23 & 0,818  \\
            Verfahrensdauer (z-std.) & -0,0458 & 0,112 & -0,41 & 0,683  \\
            Dauer $\times$ Frankreich & 0,1962 & 0,117 & 1,68 & 0,093  \\
            Dauer $\times$ Deutschland & 0,0685 & 0,117 & 0,59 & 0,558  \\
            Auftragswert (log, z-std.) & 0,0254 & 0,018 & 1,42 & 0,155  \\
            Verfahrensart (Selektiv) & -0,0008 & 0,037 & -0,02 & 0,984  \\
            \addlinespace
            \textit{Zeit-Fixed-Effects} & & & & \\
            Jahr: 2015 bis 2022 & \multicolumn{4}{c}{Inkludiert (Alle signifikant $p < 0,001$)}  \\
            \midrule
            Beobachtungen (n) & 77.820 & & &  \\
            \bottomrule
        \end{tabular*}
        \begin{tablenotes}[flushleft, para]
            \vspace{0.1cm}
            \footnotesize
            \textit{Anmerkung:} Modell nur für $procurement\_category = 'services'$. * $p < 0,05$, *** $p < 0,01$.
        \end{tablenotes}
    \end{threeparttable}\label{tab:table4}
\end{table}

\newpage
% --- Tabelle A5: Code-Verfügbarkeit ---
\begin{table}[htbp]
    \centering
    \begin{threeparttable}
        \caption{Digitale Anhänge und Code-Verfügbarkeit}
        \label{tab:code_availability}
        \begin{tabular*}{\textwidth}{l @{\extracolsep{\fill}} l}
            \toprule
            \multicolumn{2}{l}{Zur Gewährleistung der Replizierbarkeit sind die Skripte online hinterlegt:} \\
            \midrule
            \textbf{Datenanalyse} & \url{https://github.com/SofiaSch/BA_Analysis} \\
            \textit{Inhalt:} & Datenbereinigung, ZINB-Modelle, Visualisierungen \\
            \addlinespace
            \textbf{Schreibskript} & \url{https://github.com/SofiaSch/Bachelorarbeit} \\
            \textit{Inhalt:} & LaTeX-Framework, Automatisierung der Tabellengenerierung \\
            \bottomrule
        \end{tabular*}
        \begin{tablenotes}[flushleft, para]
            \vspace{0.1cm}
            \footnotesize
            \textit{Anmerkung:} Die Repositories enthalten eine README-Datei und eine \texttt{requirements.txt}.
        \end{tablenotes}
    \end{threeparttable}\label{tab:table5}
\end{table}