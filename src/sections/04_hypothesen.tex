\newpage
\section{Hypothesen}\label{sec:hypothesen}

Ausgehend von der zentralen Forschungsfrage, ob die Verfahrensdauer Einfluss auf den Wettbewerb und die KMU-Beteiligung in Deutschland, Frankreich und Estland hat, lassen sich drei Hypothesen ableiten.

\subsection{Der Einfluss der Verfahrensdauer auf die Wettbewerbsintensität}\label{subsec:der-einfluss-der-verfahrensdauer-auf-die-wettbewerbsintensitat}

Die erste Hypothese beruht auf der Annahme, dass eine längere geplante Verfahrensdauer zu höheren Transaktionskosten führt.
Das umfasst den administrativen Aufwand, sowie die damit einhergehenden Kapitalbindungen.
Für Unternehmen kann eine langwierige Dauer als Indikator für einen komplexeren, aufwendigeren und potenziell unsicheren Vergabeprozess erscheinen.
Die hohen Kosten und das vermeintliche Risiko führen dazu, dass Unternehmer von einem Angebot absehen, was zu geringerer Wettbewerbsintensität beiträgt.

\textit{H1: Je länger die geplante Verfahrensdauer einer öffentlichen Ausschreibung ist, desto geringer ist die Anzahl der abgegebenen Gebote.}

\subsection{Der Einfluss der Verfahrensdauer auf die KMU-Beteiligung}\label{subsec:der-einfluss-der-verfahrensdauer-auf-die-kmu-beteiligung}

Anknüpfend an die Argumentation der ersten Hypothese, wonach lange Verfahrensdauern als Kostentreiber und Unsicherheitsfaktor wirken, wird vermutet, dass KMU hiervon überproportional stark betroffen sind.
Aufgrund ihrer begrenzteren personellen und finanziellen Ressourcenkapazitäten stellt die Bindung von Mitteln über einen langen Zeitraum für KMU eine signifikant höhere Hürde dar als für Großunternehmen.
Es ist daher anzunehmen, dass lange Verfahrensdauern spezifisch kleinere Bieter abschrecken und somit deren Anteil an den eingereichten Geboten sinkt.

\textit{H2: Je länger die geplante Verfahrensdauer ist, desto geringer ist der Anteil der Gebote kleiner und mittlerer Unternehmen.}

\subsection{Der moderierende Einfluss des Länderkontexts}\label{subsec:der-moderierende-einfluss-des-landerkontexts}

Die abschreckende Wirkung der langen Verfahrensdauer hängt von der wahrgenommenen Effizienz der nationalen Verwaltung ab.
Aufgrund der komplexen föderalen Strukturen und des Rufs einer hohen Regelungsdichte wird der abschreckende Effekt in Deutschland am größten erwartet.
Frankreich wird ebenfalls bürokratisch, aber zentralisierter eingeschätzt, während im hoch digitalisierten Estland eine lange Dauer am ehesten als sachlich begründet akzeptiert wird.

\textit{H3: Der negative Einfluss der geplanten Verfahrensdauer auf den Wettbewerb ist in Deutschland am stärksten, in Frankreich moderat und in Estland am schwächsten ausgeprägt.}