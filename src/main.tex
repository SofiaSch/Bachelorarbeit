% =============================================================================
% HAUPTDOKUMENT (main.tex)
% =============================================================================

\documentclass[12pt, a4paper, ngerman, bibliography=totoc]{scrartcl}

% --- GRUNDEINSTELLUNGEN UND PAKETE ---
\usepackage[utf8]{inputenc}
\usepackage[T1]{fontenc}
\usepackage[ngerman]{babel}
\usepackage{times} % Schriftart Times New Roman

% --- SEITENLAYOUT NACH VORGABE ---
\usepackage[margin=1in]{geometry} % 1 Zoll an allen Rändern
\usepackage{setspace}
\doublespacing % 2.0-facher Zeilenabstand

% --- PAKETE FÜR INHALTE ---
\usepackage{graphicx}       % Für Bilder
\usepackage{booktabs}       % Für schönere Tabellen (wichtig für APA)
\usepackage{tabularx}
\usepackage{array}
\usepackage{xcolor}
\usepackage{float}
\usepackage{amsmath}
\usepackage{threeparttable}
\usepackage{xurl}
\usepackage{acronym}

% --- \tabcite Befehl ---
\definecolor{meingrau}{gray}{0.6} % Definiert ein mittleres Grau

\usepackage{caption}
\captionsetup[table]{
    format=plain,
    singlelinecheck=false,      % Erzwingt Links-Bündigkeit auch bei kurzen Titeln
    justification=raggedright,  % Links ausgerichtet
    labelsep=newline,           % Zeilenumbruch nach "Tabelle 1"
    font={stretch=1},           % Zeilenabstand normal in der Caption
    labelfont=bf,               % "Tabelle 1" fett
    textfont=it                 % Titeltext kursiv
}

% \tabcite: Erzeugt Zeilenumbruch, setzt Farbe und Link-Farbe auf Grau
\newcommand{\tabcite}[1]{%
    \newline
    {%
        \footnotesize           % Schriftgröße klein
        \color{meingrau}        % Färbt Klammern und Kommas
        \hypersetup{citecolor=meingrau} % WICHTIG: Färbt den Link (die Jahreszahl)
        \parencite{#1}%
    }%
}
% --- Ende Befehl ---

\usepackage[
    backend=biber,
    style=apa,
    sorting=nyt
]{biblatex}                 % FÜR ZITATIONEN UND LITERATURVERZEICHNIS (EINZIGES PAKET!)
\usepackage{csquotes}

\addbibresource{bibliothek.bib} % Verknüpfung zur Literaturdatenbank
\definecolor{TUMBlau}{HTML}{005293}


\usepackage{hyperref}
\usepackage{amsopn}
\usepackage{eurosym}
\hypersetup{
    colorlinks=true,
    linkcolor=black,
    citecolor=black,
    urlcolor=blue
}

% =============================================================================
% DOKUMENTENBEGINN
% =============================================================================
\begin{document}

% --- TITELSEITE ---
% =============================================================================
% DECKBLATT (00_cover.tex) - Bereinigte Version
% =============================================================================
% Diese Datei enthält NUR den Inhalt der Titelseite.
% Alle Pakete und Dokumenteneinstellungen werden in 'main.tex' geladen.

\begin{titlepage}
\begin{singlespacing} %

    \pagestyle{empty} % Keine Seitenzahl auf dem Deckblatt
    \vspace*{-2.5cm} % Leichte Anpassung für den oberen Rand

    % --- KOPFZEILE MIT LEHRSTUHL-INFOS UND LOGO ---
    \begin{minipage}[t]{0.7\textwidth}
        \raggedright % Text linksbündig
        \footnotesize\color{TUMBlau} % Farbe und Schriftgröße aus main.tex
        Chair for Strategy and Organization \\
        TUM School of Management \\
        Technical University of Munich
    \end{minipage}%
    \hfill % Flexibler Abstand zwischen den Boxen
    \begin{minipage}[t]{0.25\textwidth} % Etwas schmaler für besseren Look
        \raggedleft % Logo rechtsbündig
        \includegraphics[width=2.5cm]{images/tum_logo}
    \end{minipage}

    \vspace{2cm}

    % --- TITELBLOCK ---
    \centering % Zentriert den gesamten Block
    \vfill % Schiebt den Titelblock vertikal in die Mitte
    {\huge\bfseries Verfahrensdauer und Wettbewerb in der öffentlichen Auftragsvergabe\par}
    \vspace{0.5cm} % Sauberer Abstand zwischen Haupt- und Untertitel
    {\Large Eine vergleichende Analyse der Effekte auf die Gesamt- und KMU-Beteiligung in Deutschland, Estland und Frankreich\par}
    \vfill % Schiebt den Rest nach unten

    % --- ART DER ARBEIT ---
    {\Large Bachelorarbeit\par}
    zur Erlangung des akademischen Grades\par
    Bachelor of Science (B.Sc.)\par
    an der Technischen Universität München\\[3cm]

    % --- PRÜFER-BLOCK UND BILD ---
    \begin{minipage}[c]{0.5\textwidth}
        \large
        \raggedright
        \textbf{Prüfer:} \par
        Prof. Dr. Isabell M. Welpe \par
        Chair of Strategy and Organization \par
        \vspace{1cm}
        \textbf{Betreuer:} \par % Gängigere Bezeichnung als "Beaufsichtigt von"
        Sebastian Bauer \par
        \vspace{1cm}
        \textbf{Eingereicht von:}\par
        Sofia Schepers\par
        Matrikelnummer: 03773506
    \end{minipage}%
    \hfill % Abstand
    \begin{minipage}[c]{0.4\textwidth}
        \centering
        \includegraphics[width=\textwidth]{images/Uhrenturm}
    \end{minipage}

    \vfill % Flexibler Abstand zum Abgabedatum

    % --- ABGABEDATUM ---
    {\large
    \textbf{Eingereicht am:}\par
    \today
    }

\end{singlespacing}
\end{titlepage}
\newpage

% --- ABSTRACT ---
\section*{Abstract}

Die Beteiligung kleiner und mittlerer Unternehmen (KMU) an der öffentlichen Auftragsvergabe ist essenziell für den Wettbewerb in der EU, wird jedoch oft durch administrative Komplexität erschwert.
Diese Bachelorarbeit untersucht den Einfluss der geplanten Verfahrensdauer auf die Wettbewerbsintensität und die KMU-Beteiligung unter Berücksichtigung des moderierenden nationalen Verwaltungskontexts.

Theoretisch fundiert durch die Transaktionskosten- und Signaltheorie, wird die Verfahrensdauer dabei als Indikator für administrative Effizienz betrachtet.
Die empirische Prüfung erfolgt mittels einer quantitativen Analyse von OpenTender.eu-Daten aus Deutschland, Frankreich und Estland unter Anwendung von ZINB- und Fractional-Logit-Regressionsmodellen.

Die Ergebnisse widerlegen eine universell abschreckende Wirkung der Verfahrensdauer (H1).
Während längere Fristen in Deutschland und Frankreich tendenziell als notwendige Vorbereitungsressource positiv mit der Gebotsanzahl korrelieren, zeigt sich in Estland ein signifikanter negativer Effekt auf die KMU-Beteiligung (H2).
Hier signalisiert eine längere Dauer administrative Anomalien, die KMU überproportional belasten.

Die Arbeit zeigt die Relevanz von Kontextfaktoren für die Transaktionskostentheorie auf.
Für die Praxis empfiehlt sich eine länderspezifische Steuerung der Fristen sowie eine stärkere Standardisierung und Zentralisierung von Vergabeprozessen, um die Verlässlichkeit administrativer Signale und den Marktzugang für KMU zu verbessern.

\begin{quote}
\textit{Kennwörter: Öffentliche Auftragsvergabe, KMU-Beteiligung, Verfahrensdauer, Transaktionskostentheorie, Signaltheorie, internationaler Vergleich.}
\end{quote}
\newpage

% --- INHALTSVERZEICHNIS ---
\tableofcontents
\newpage

% --- HAUPTTEIL DER ARBEIT ---
% Die Struktur hier folgt der idealen Gliederung für empirische Arbeiten
\section{Einleitung}
% Die Einleitung folgt dem "umgedrehten Pyramiden"-Prinzip[cite: 2484].
% 1. Relevanz des Themas (Praxis und/oder Wissenschaft)[cite: 2427].
% 2. Forschungsbedarf aufzeigen (Lücke, widersprüchliche Ergebnisse, methodische Mängel)[cite: 2430].
% 3. Ziel der Arbeit klar formulieren[cite: 2432].
% 4. Theoretischer Beitrag (Welche Wissenslücke wird geschlossen?)[cite: 2433].
% 5. Praktische Implikationen (Für wen sind die Ergebnisse relevant?)[cite: 2437].
% 6. Kurzer Ausblick auf die Vorgehensweise[cite: 2439].
\newpage
\section{Theoretischer Hintergrund}\label{sec:theoretischer-hintergrund}

Die Beantwortung der zentralen Forschungsfrage, welche Auswirkungen die Verfahrensdauer auf die Wettbewerbsintensität und die Beteiligung von KMUs hat, bildet den Gegenstand dieses Kapitels.
Zu diesem Zweck wird zunächst der theoretische Hintergrund erörtert.
In der vorliegenden Arbeit werden nach einer Darlegung der allgemeinen Informationen über das Vergabeverfahren die spezifischen Aspekte des Themas beleuchtet, woraus sich schließlich die zu prüfenden Hypothesen ergeben.

\subsection{Die Grundlagen der öffentlichen Auftragsvergabe}\label{subsec:die-grundlagen-der-offentlichen-auftragsvergabe}

Die öffentliche Auftragsvergabe stellt einen der bedeutendsten Wirtschaftsfaktoren moderner Volkswirtschaften dar.
Mit einem Anteil von etwa 12\% am Bruttoinlandsprodukt (BIP) und fast einem Drittel der gesamten Staatsausgaben in den OECD-Ländern werden immense Summen in Infrastruktur, Güter und Dienstleistungen investiert~\parencite{Value_Creation_in_PP}.
Allein in Deutschland flossen im Jahr 2019 rund 35\% der Staatsausgaben über diesen Kanal in die Wirtschaft, wobei Länder und Kommunen die größten Auftraggeber sind~\parencite{Oeffentliche_Auftraege_DE}.

Doch was genau verbirgt sich hinter dem Begriff des öffentlichen Vergabeverfahrens?
Im Allgemeinen handelt es sich um einen Vertrag zwischen der öffentlichen Hand und Unternehmen über die Erbringung von Liefer-, Dienst- oder Bauleistungen.
Als öffentliche Auftragsgeber werden alle Dienstellen des Bundes, der Länder, Gemeindeverbände und sonstige juristische Personen des öffentlichen Rechts, wie beispielsweise Hochschulen, definiert.
Außerdem Einrichtungen, die vom Staat finanziert werden (z.B.\ Krankenhäuser) und weitere Besonderheiten~\parencite{Oeffentliche_Auftraege_DE}.

Im Jahr 2014 wurde ein großer Schritt bezüglich des europäischen Vergaberechts gemacht, da dort neue EU-Vergaberichtlinien verabschiedet wurden~\parencite{Oeffentliche_Auftraege_DE}.
Diese zielen auf eine Vereinfachung der Verfahren und Stärkung der KMU-Beteiligung ab~\parencite{EU2014RL24}.
Dadurch haben sich grundlegende Prinzipien für das Vergabeverfahren etabliert, die das Ziel verfolgen das objektiv beste Angebot mit einem optimalen Preis-Leistung-Verhältnis auszuwählen.
Besonders relevant, auch im Bezug auf die KMU, gelten die 3 Prinzipien: Gleichheit, Nichtdiskriminierung und Transparenz~\parencite{JIM_Fair_Transparent_Competitive}.
Sie sollen garantieren, dass alle Unternehmen die gleiche Chance haben und niemand bevorzugt wird.
Als zentrales Prinzip gilt der Grundsatz des Wettbewerbs, welcher dazu führt, dass alle anderen Prinzipien erreicht werden~\parencite{Oeffentliche_Auftraege_DE}.
Um zudem das ökonomische Ziel zu erfüllen, gilt das Prinzip der Effizienz beziehungsweise Wirtschaftlichkeit~\parencite{JIM_Fair_Transparent_Competitive}.
Dieses Prinzip gewährleistet, dass hinsichtlich der eingesetzten Steuergelder das wirtschaftlichste und finanziell beste Resultat erzielt wird.

Diese Prinzipien bilden die normative Grundlage, auf der das gesamte System der öffentlichen Auftragsvergabe in der EU beruht.

Ein weiteres wichtiges Instrument im Vergaberecht sind die Schwellenwerte.
Diese werden von der EU-Kommission festgelegt und definieren, ob ein Auftrag national oder europaweit ausgeschrieben werden muss.
Bei sehr geringen Auftragswerten ist keine Ausschreibung notwendig, um den bürokratischen Aufwand gering zu halten.
Die Höhe der Schwellenwerte unterscheidet sich je nach Art des Auftrags~\parencite{schwellenwerte}.

In der Auftragsvergabe gibt es mehrere Verfahrensarten.
Der Unterschied besteht im Wesentlichen in ihrer Offenheit und dem Grad des Wettbewerbs.
In der öffentlichen Ausschreibung (beziehungsweise dem offenen Verfahren) werden eine unbegrenzte Anzahl an Unternehmen bekannt gegeben.
Dadurch wird ein uneingeschränkter Wettbewerb garantiert und somit das wirtschaftlich wertvollste Angebot ermittelt.
Alternativ gibt es die beschränkte Ausschreibung (beziehungsweise das nicht offene Verfahren).
Die Anzahl der Unternehmen ist beschränkt, welche von dem Auftraggeber angesprochen werden.
Zusätzlich gibt es noch die Verhandlungsvergabe, wodurch die Auftraggeber den meisten Freiraum haben.
Hier wird direkt mit ausgewählten Unternehmen verhandelt, sie ist jedoch nur unter bestimmten Voraussetzungen zulässig.
Unabhängig von der gewählten Art folgt der Verfahrensablauf einem strukturierten, mehrstufigen Prozess~\parencite{Oeffentliche_Auftraege_DE}.

Der grobe Verfahrensablauf ist in mehrere Schritte unterteilt, wie Abbildung~\ref{fig:verfahrensablauf} veranschaulicht.
Im ersten Schritt erfolgt die Auftragsbekanntmachung.
Je nachdem, ob es sich um ein nationales oder ein europaweites Verfahren handelt, können die Unterlagen entweder direkt heruntergeladen oder angefordert werden.
Anschließend können die Teilnahmeanträge eingereicht und Angebote abgegeben werden.
Die Angebote werden geprüft und bewertet, anschließend wird die Zuschlagserteilung bekannt gegeben.

\begin{figure}[h!]
    \centering
    \includegraphics[width=0.8\textwidth]{images/verfahrensablauf}
    \caption{Der grobe Ablauf des öffentlichen Vergabeverfahrens.}
    \label{fig:verfahrensablauf}
    \footnotesize
    \parencite{Oeffentliche_Auftraege_DE}
\end{figure}

\subsection{Bürokratie als Transaktionskostentreiber}\label{subsec:burokratie-als-transaktionskostentreiber-in-der-offentlichen-vergabe}

Damit die Grundprinzipien wie Transparenz, Wettbewerb und Gleichbehandlung eingehalten werden können erfordert das formalisierte und standardisierte Prozesse.
Diese erfolgen in Regeln, Dokumentationspflichten und mehrstufigen Abläufen und bilden somit die Bürokratie des Vergabewesens.
Was zweifelsfrei notwendig ist, kann dennoch eine Hürde sein, da eine hohe Bürokratie mit hohen Kosten verbunden ist~\parencite{Discretion_supplier_selection}.
In diesem Fall werden die Kosten nicht zwangsläufig in monetären Werten berechnet, sondern in Form von Transaktionskosten.
Das sind alle Kosten, die beim Abschluss eines Geschäfts entstehen~\parencite{transaktionskosten}.
Darunter fallen die direkten Kosten, die sofort sichtbar und messbar sind.
Wie auch die indirekten Kosten, wie zum Beispiel die Such- und Informationskosten oder Verhandlungs- und Entscheidungskosten~\parencite{transaktionskosten_2}.
Zusammengefasst also alle Aufwände, die ein Unternehmen hat, um überhaupt an der Ausschreibung teilnehmen zu können.

In einer Ausschreibung fallen direkte Kosten in Form von Ressourcenbindung an.
Dazu gehört die Kapitalbindung, was insbesondere die Opportunitätskosten erhöht, da dieses Kapital nicht für andere Projekte genutzt werden kann~\parencite{opportunitaetskosten}.
Außerdem die Personalbindung, da sich wichtige Mitarbeiter mit der Ausschreibung befassen und ihre Kapazitäten in das ausarbeiten des Angebots stecken müssen.

Indirekte Kosten treten durch die Signalwirkung auf.
Eine lange Frist gilt oft als Signal für Komplexität, wodurch der Prozess als ineffizient und schwerfällig interpretiert wird.
Zudem kann eine lange Frist für Unsicherheit, sowohl auf der Angebots- als auch auf der Nachfrageseite - je länger der Prozess, desto größer das Risiko, dass sich Marktbedingungen, Kosten oder Anforderungen ändern~\parencite{PPP_transparency}.

Bürokratie verursacht Transaktionskosten, woraus sich die logische Konsequenz ergibt, dass bei steigenden Kosten und Risiken der erwartete Nettonutten für ein Unternehmen sinkt.
Dementsprechend werden sich Unternehmen gegen eine Angebotsabgabe von übermäßig langen Verfahren entscheiden~\parencite{Discretion_supplier_selection}.

Während diese steigenden Transaktionskosten alle Unternehmen betreffen, sind kleine und mittlere Unternehmen aufgrund ihrer spezifischen strukturellen Merkmale davon in besonderem Maße betroffen~\parencite{PPP_transparency}.

\subsection{Die Sondersituationen von kleinen und mittleren Unternehmen (KMU)}\label{subsec:die-sondersituationen-von-kleinen-und-mittleren-unternehmen-(kmu)}

Im Jahr 2023 zählten alleine 99\% aller Unternehmen in Deutschland zu der Gruppe der kleinen und mittleren Unternehmen.
Das umfasst ein Beschäftigungsvolumen von etwa 53\% und 41\% der Bruttowertschöpfung~\parencite{KMU_numbers}.
Die Zahlen alleine zeigen bereits, dass KMU ein großer und wichtiger Teil der Wirtschaft sind.
Und auch die Erleichterungen durch die Reform zeigen, dass der Staat die relevanz erkannt hat:

\begin{blockquote}
    Die öffentliche Vergabe sollte an die Bedürfnisse von KMU angepasst werden.~\parencite[Erwägungsgrund 78]{EU2014RL24}
\end{blockquote}

So sieht die neue Reform vor, dass große Aufträge in kleinere Teil- und Fachlose aufgeteilt wird, sodass diese auch von kleinen Unternehmen erfüllt werden können~\parencite[Artikel 46]{EU2014RL24}.
Eine andere Maßnahme ist, dass das Unternehmen keine unverhältnismäßig großen Jahresumsatz vorweisen muss, um an der Ausschreibung teilzunehmen.
Seit der Reform darf in der Regel nur noch das Zweifache des Auftragswertes gefordert werden~\parencite[Artikel 58, Absatz 3]{EU2014RL24}.

Obwohl die Richtlinien das Vergabeverfahren für KMU geöffnet hat, besteht weiterhin Hürden, die die Teilnahme erschwert.
Kapital und Personal ist begrenzt und besonders junge und sehr kleine Unternehmen verfügen über wenig Ressourcen, die sie für komplexe Ausschreibungen zur Verfügung haben.
Auch zu externem Kapital ist der Zugang meist begrenzt~\parencite{SME_Instruments}.
Das kann bedeuten, dass KMU weder die Zeit noch das Kapital haben um neben dem Tagesgeschäft weitere Ressourcen für Ausschreibungen zu entbehren.

Ein weiterer Punkt ist die geringere Erfahrung im Umgang mit übermäßig bürokratischen Prozessen~\parencite{JIM_Fair_Transparent_Competitive}.
Das führt zu erhöhter Unsicherheit sich überhaupt erst zu bewerben oder durch formale Fehler ausgeschlossen zu werden.

Diese Kombination von geringeren Ressourcen und erhöhten Ressourcen kann zur folge haben, dass bei einem langen Verfahren mit hohen Transaktionskosten der Abschreckungseffekt überproportional hoch ist.
Der Effekt tritt bei KMU stärker auf, trotz der neuen Reform, was dazu führt, das diese aus dem Wettbewerb verdrängt werden~\parencite{PPP_transparency}.




\subsection{Der Länderspezifische Verwaltungskontext als Moderator}\label{subsec:der-landerspezifische-verwaltungskontext-als-moderator}

Das öffentliche Vergabewesen in der Europäischen Union wird maßgeblich durch Richtlinien gerahmt, die eine weitgehende Harmonisierung der Verfahren für Aufträge oberhalb der EU-Schwellenwerte anstreben.
Da der für diese Analyse verwendete Datensatz primär solche oberschwelligen Verfahren umfasst, gelten für Deutschland, Frankreich und Estland theoretisch dieselben rechtlichen Rahmenbedingungen hinsichtlich Fristen und Bekanntmachungen.
Dennoch lassen sich signifikante Unterschiede in der Praxis der Auftragsvergabe erwarten.
Diese resultieren nicht aus unterschiedlichen Gesetzen, sondern aus der administrativen Umsetzung und der gelebten Verwaltungskultur innerhalb der Mitgliedsstaaten.
Nationale Unterschiede in der Behördenstruktur – etwa zentralisiert versus föderal – und dem Digitalisierungsgrad beeinflussen entscheidend, wie effizient die EU-Vorgaben in die Praxis umgesetzt werden.
Dies bestimmt maßgeblich, wie hoch die tatsächlichen Transaktionskosten für Bieter ausfallen und wie stark diese den Wettbewerb beeinträchtigen.


\subsubsection{Deutschland: Föderale Komplexität und Regelungsdichte}

Deutschlandweit existieren rund 30.000 Vergabestellen, die sich auf Bundes-, Landes- oder kommunaler Ebene befinden~\parencite{buerokratie_DE}.
Diese hohe Zahl an dezentralen Akteuren führt zu einer heterogenen Verwaltungslandschaft.
Je nach Standort sind die Vergabestellen unterschiedlich ausgestattet, was Finanzen und Personal betrifft, obwohl von ihnen erwartet wird, dass sie sich mit allen Arten von Beschaffungen auskennen.

Diese fragmentierte Struktur ist gerade für den hier untersuchten Oberschwellenbereich von zentraler Bedeutung.
Im Gegensatz zu zentralisierten Systemen obliegt die Durchführung komplexer EU-weiter Ausschreibungen in Deutschland weiterhin oft den dezentralen Bedarfsstellen vor Ort.
Dies führt zu einem strukturellen Defizit an Professionalisierung und Routine: Viele kleinere Vergabestellen müssen die hochkomplexen Anforderungen des EU-Vergaberechts umsetzen, ohne über die spezialisierten Fachkapazitäten zentraler Beschaffungsbehörden zu verfügen.~\parencite{buerokratie_DE}.

Das föderale System Deutschlands verstärkt diesen Effekt.
Es gilt das Grundprinzip der Zuständigkeit, wonach der Verwaltungsvollzug in der Regel Ländersache ist.
Dies berechtigt die Bundesländer eigene Organisations- und Verfahrensregeln aufzustellen, was unter anderem zu einer Zersplitterung der E-Vergabe-Landschaft führt.
Hinzu kommt das Recht der Kommunen auf Selbstverwaltung, wodurch diese ihre Verfahren ebenfalls eigenständig regeln können.
Für bietende Unternehmen hat dies zur Folge, dass sie sich trotz harmonischeren EU-Rechts je nach ausschreibender Kommune und Bundesland immer wieder mit unterschiedlichen Abläufen, technischen Plattformen und Bearbeitungsroutinen auseinandersetzen müssen~\parencite{aufgaben_foederalstaat}.

\subsubsection{Frankreich: Zentralistische Tradition}

Deutschland als Bundesstaat und Frankreich als Einheitsstaat repräsentieren zwei gegensätzliche administrative Modelle.
Während Deutschland durch seinen ausgeprägten Föderalismus geprägt ist, zeichnet sich Frankreich durch eine historisch gewachsene, starke Zentralisierung aus.
Dieser strukturelle Unterschied wirkt sich direkt auf die Umsetzung von EU-Vergabeverfahren aus.
Zwar unterliegen beide Länder denselben EU-Richtlinien, doch profitiert Frankreich von einer wesentlich höheren Standardisierung der Abläufe.
Mit dem Code de la commande publique existiert ein einheitliches Gesetzbuch, das die Vergaberegeln landesweit bündelt und für Bieter transparenter macht.

Ein entscheidender Vorteil gegenüber der deutschen Fragmentierung ist zudem die Existenz starker zentraler Beschaffungsstellen, insbesondere der UGAP (Union des groupements d’achats publics).
Diese Institution bündelt Beschaffungsvolumina und schließt komplexe Rahmenverträge zentral ab.
Für Unternehmen bedeutet dies, dass sie es bei großen Ausschreibungen häufig mit hochprofessionellen, zentralen Ansprechpartnern zu tun haben, statt mit einer Vielzahl kleiner, lokaler Vergabestellen.

Diese administrative Professionalisierung korrespondiert mit einer höheren Akzeptanz digitaler Prozesse.
Laut dem Index für digitale Wirtschaft und Gesellschaft (DESI) 2022 der Europäischen Kommission nutzen rund 87\% der französischen Internetnutzer elektronische Behördendienste, während dieser Wert in Deutschland bei lediglich 55\% liegt.
Die Kombination aus zentraler Steuerung und höherer digitaler Affinität lässt für Frankreich niedrigere Transaktionskosten erwarten als im föderalen Deutschland~\parencite{GovTech_FR}.

\subsubsection{Estland: Digitale Vorreiterschaft}

Mit lediglich 1,3 Millionen Einwohnern (Stand: 2024) ist Estland deutlich kleiner als Frankreich und Deutschland.
Dennoch betreibt Estland das wohl umfassendste digitale Regierungssystem der Welt~\parencite{estonia_powerhouse}.
Es ist das erste Land, welches 100\% ihrer Regierungsdienstleistungen digitalisiert hat.
So können die Bürger ein Unternehmen in unter 15 Minuten anmelden, wählen und sich scheiden lassen, ohne das Haus zu verlassen~\parencite{estonia_powerhouse}.
Diese Entwicklung began bereits im Jahr 1999, als sich Estland dazu entschieden hat nicht die bestehenden Systeme zu nutzen sondern eigene Systeme von Grund auf neu aufzubauen~\parencite{estonia_digital}.

Zwei wesentliche Säulen tragen dabei eine wesentliche Rolle.
\begin{itemize}
\item\textbf{Die nationale digitale ID (e-ID):} Dadurch haben Bürger die Möglichkeit sich digital zu verifizieren und online zu unterschreiben.

\item\textbf{Die X-Road:} Eine interoperable Plattform zum Datenaustausch.
Dadurch können getrennte Datenbanken sicher und standardisiert miteinander kommunizieren.
\end{itemize}

Dazu kommt das \("\)Once-Only\("\)-Prinzip.
Dieses ist gesetzlich verankert und verpflichtet alle Behörden die X-Road zu nutzen.
Dadurch haben die Bürger eine zentrale Stelle, an welcher alle Informationen gesammelt und aktualisiert werden.
Das Vergabesystem kommuniziert somit direkt mit anderen staatlichen Registern\textsuperscript{}.
Für Bieter bedeutet dies, dass die formale Komplexität einer EU-Ausschreibung durch Automatisierung maskiert wird.~\parencite{estonia_powerhouse}.

Diese Maßnahmen haben einen großen Einfluss, nicht nur auf die Bürger, sondern auch auf die Wirtschaft.
Schätzungsweise spart das System der Gesellschaft bis zu 1.400 Arbeitsstunden jährlich und hat einen Einfluss von 4-7\% des jährlichen Bruttoinlandproduktes~\parencite{estonia_powerhouse}.

Dieses hohe Maß an institutionalisierter Effizienz und der Digitalisierung positioniert Estland als einen "Best-Practice"-Fall im europäischen E-Government, dessen administrative Logik sich fundamental von der Deutschlands und Frankreichs unterscheidet.

\subsubsection{Zusammenfassender Vergleich und Implikationen für den Wettbewerb}

Die Analyse der drei Länderkontexte offenbart fundamental unterschiedliche administrative Grundstrukturen, die sich direkt auf die wahrgenommenen Transaktionskosten und die Effizienz des Vergabewesens auswirken.
Die geplante Verfahrensdauer fungiert als Signal an die Unternehmen.
Wie dieses Signal interpretiert wird – ob als Indikator für sachliche Komplexität oder für administrative Ineffizienz – hängt vom Vertrauen in das jeweilige Verwaltungssystem ab.

In Deutschland treffen Unternehmen auf ein hochgradig föderales und fragmentiertes System.
Die Existenz von rund 30.000 Vergabestellen führt dazu, dass selbst bei harmonisiertem EU-Recht die administrative Umsetzung extrem heterogen ist.
Für viele kleinere Stellen sind komplexe EU-Verfahren keine Routine, was zu Unsicherheiten und ineffizienten Abläufen führt.
Eine lange Verfahrensdauer dürfte hier primär als Signal für hohe, unkalkulierbare Bürokratie und mangelnde Professionalisierung interpretiert werden.

In Frankreich agieren Unternehmen in einem zentralistischen System, das durch den Code de la commande publique und starke zentrale Beschaffungsstellen (UGAP) geprägt ist.
Diese Professionalisierung sorgt für eine höhere Standardisierung der Abläufe.
Gepaart mit einer signifikant höheren Akzeptanz digitaler Behördendienste (87 \% Nutzung), erscheint das System für Bieter berechenbarer.
Eine lange Dauer wird hier eher als notwendige Prüfzeit einer professionellen Bürokratie verstanden denn als Ineffizienz.

In Estland trifft die Verfahrensdauer auf ein „Digital-by-Default“-System, das auf maximale Effizienz ausgelegt ist.
Durch das gesetzlich verankerte \enquote{Once-Only}-Prinzip und die X-Road-Infrastruktur werden die Transaktionskosten für Bieter systemisch minimiert, da Nachweise automatisch abgerufen werden.
Eine lange Verfahrensdauer ist in diesem hocheffizienten Umfeld eine Anomalie und signalisiert am ehesten eine hohe inhaltliche Komplexität des Auftrags.

Die in Tabelle 1 zusammengefassten strukturellen Unterschiede stützen die Annahme, dass die abschreckende Wirkung langer Verfahrensdauern kein universelles Phänomen ist, sondern vom nationalen Verwaltungskontext moderiert wird.
Aus diesen theoretischen Vorüberlegungen leitet sich die dritte Hypothese ab.



\begin{table}[H]
\centering
\caption{Administrativ-struktureller Ländervergleich}
\label{tab:admin_vergleich}

\renewcommand{\arraystretch}{1.5} % Erhöht den Zeilenabstand für bessere Lesbarkeit

% --- SPALTENDEFINITION ---
\begin{tabularx}{\textwidth}{@{} >{\raggedright\arraybackslash}p{3.5cm} >{\raggedright\arraybackslash}X >{\raggedright\arraybackslash}X >{\raggedright\arraybackslash}X @{}}
\toprule
\textbf{Merkmal} & \textbf{Deutschland} & \textbf{Frankreich} & \textbf{Estland} \\
\midrule

Administrative Grundstruktur &
Föderal \& dezentral \tabcite{aufgaben_foederalstaat} &
Zentralistischer Einheitsstaat \tabcite{unterschied_FR_DE} &
\enquote{Digital-by-Default}-Staat \tabcite{estonia_powerhouse} \\

Umsetzung \& Prozess-Standardisierung &
Dezentral \& fragmentiert (Mangelnde Routine bei EU-Verfahren) \tabcite{aufgaben_foederalstaat} &
Zentralisiert \& professionalisiert (Starke Bündelung) \tabcite{GovTech_FR} &
Vollständig automatisiert \& datengetrieben \tabcite{estonia_digital} \\

Bündelung \& Effizienz-Prinzipien &
Fragmentiert (ca. 30.000 Stellen) \tabcite{buerokratie_DE} &
Zentral gebündelt (UGAP) \tabcite{GovTech_FR} &
Systemisch (Gesetzl. \enquote{Once-Only}-Prinzip) \tabcite{estonia_powerhouse} \\

Digitale Nutzung (Bürger/ Unternehmen) &
Gering (55\% Nutzer) \tabcite{GovTech_FR} &
Hoch (87\% Nutzer) \tabcite{GovTech_FR} &
Vollständig (99\% e-ID) \tabcite{estonia_powerhouse} \\
\bottomrule
\end{tabularx}

% --- QUELLE / ANMERKUNG (APA-Stil: Linksbündig unter der Tabelle) ---
\vspace{1mm} % Kleiner Abstand
\begin{minipage}{\textwidth} % Hilft, den Text genau so breit wie die Tabelle zu halten
    \footnotesize \textit{Anmerkung.} Eigene Darstellung basierend auf den zitierten Quellen.
\end{minipage}

\end{table}
\clearpage
%! Author = sofia_privat
%! Date = 30.09.25

% Preamble
\documentclass[11pt]{article}

% Packages
\usepackage{amsmath}

% Document
\begin{document}



\end{document}
\newpage
\section{Methodik}\label{sec:methodik}

Um die Hypothesen zu überprüfen und somit die Forschungsfrage zu beantworten führen wir ein quantitatives, komparatives Ex-post-facto-Forschungsdesign durch.
Die Hypothese wird mittels statistischer Regressionsanalysen auf Basis numerischer Daten getestet und ist somit quantitativ.
Der gezielte Vergleich der drei Länder Deutschland, Frankreich und Estland nach dem \("\)Most Different System Design\("\) (MDSD) stellt eine komparative Analyse dar.
Zudem beziehen wir uns in der Analyse auf bereits vergangene Vergabeverfahren die nicht experimentell manipuliert wurden, was das Design als Ex-post-facto-Ansatz ausweist.

\subsection{Datensatz und Stichprobe}\label{subsec:datensatz-und-stichprobe}

\subsubsection{Datenquelle und -gewinnung}

Die Daten, die für die empirische Analyse genutzt werden stammen aus der OpenTender Plattform, welche vom Government Transparency Institute gesammelt, transformiert und veröffentlicht wird~\parencite{GTI2024}.
In der Datenbank befindet sich die Daten von insgesamt 35 Gerichtsbarkeiten, davon 27 Mitgliedstaaten der Europäischen Union~\parencite{GTI2024}.
Es handelt sich um einen Sekundärdatensatz.

Die Daten werden unter anderem im JSON Format zur Verfügung gestellt.
Der Untersuchungszeitraum ist auf die Jahre von 2014 bis 2022 festgelegt.
In 2014 wurde ein großer Schritt bezüglich des europäischen Vergaberechts gemacht, da dort neue EU-Vergaberichtlinien verabschiedet wurden.
Diese zielen auf eine Vereinfachung der Verfahren und Stärkung der KMU-Beteiligung ab~\parencite{EU2014RL24}.
Die Wahl für das Jahr 2022 beruht auf der Datenverfügbarkeit, da das neusten Daten für Frankreich aus dem Jahre 2022 sind.
Die Analyse dieses Zeitraums ermöglicht die Analyse innerhalb des aktuellen rechtlichen Rahmens auf einer gemeinsamen Grundlage.

\subsubsection{Länderauswahl}

Die Auswahl der Länder basiert auf dem "Most Different Systems Design" (MDSD).
Ziel ist es dabei drei Länder, innerhalb des gemeinsamen rechtlichen Rahmens der EU, mit unterschiedlichen Verwaltungsstrukturen zu vergleichen.

Deutschland zählt als Vorreiter in der Umsetzung der neuen EU-Vergaberichtlinien und ist zudem hochgradig föderalistisch und regulierungsintensiv~\parencite{paper1}.
Die dezentrale Vergabe erfolgt auf Bundes-, Landes- und Kommunalebene, was zu einer heterogenen Verwaltungslandschaft führt~\parencite{DE_FOER}. \n
Frankreich dient als gegensatz mit einem historisch zentralem Staat~\parencite{FR_Zentr}.
Die Vergaben sind größtenteils landesweit standardisiert
Als drittes Land ist Estland deutlich kleiner als die beiden wirtschaftlich starken Länder Deutschland und Frankreich.
Dennoch gilt Estland als digitaler Vorreiter, was die Digitalisierung in Europa betrifft~\parencite{EST_Digital}.

Durch den gezielten Vergleich dieser unterschiedlichen Systeme wird es möglich, über eine reine Effektmessung hinauszugehen und zu einem tieferen, kausalen Verständnis darüber zu gelangen, warum und unter welchen Bedingungen die Dauer von Vergabeverfahren den Wettbewerb tatsächlich beeinflusst.

\subsubsection{Stichprobenziehung}

Die Analyse der Daten erfordert zunächst deren Umwandlung in ein geeignetes Format, welches die Datenaufbereitung für die nachfolgende Analyse erleichtert.
Mittels eines Python-Skripts erfolgte eine Analyse aller Einträge pro Land, bei der ausschließlich die Variablen extrahiert wurden, die für die Forschungsfrage von Relevanz sind.
Im Rahmen der weiteren Analyse erfolgte eine Konvertierung der Daten in eine CSV-Datei pro Land.
Daraufhin folgt die Bereinigung.

Im ersten Schritt der Bereinigung wurde die Datumsspalte von einem Text in ein Datums-Format konvertiert.
Nachfolgend wurden sämtliche Einträge eliminiert, bei denen das Start- oder Enddatum nicht angegeben war, da diese für die Analyse nicht relevant waren.
Bei einigen Einträgen treten Logikfehler auf, die sich dadurch äußern, dass das Enddatum vor dem Startdatum liegt.
Auch diese wurden aus den betreffenden Dateien entfernt.
Anschließend wurde eine Filterung der Einträge vorgenommen, die auch im wettbewerblichen Verfahren partizipiert haben, um eine Ausschließung von Direktvergaben zu gewährleisten.

Durch die Implementierung dieser Schritte wurde eine Reduktion der Datenmenge von 1.547.352 Einträgen (Deutschland: 402.358, Frankreich: 1.097.001, Estland: 47.993) auf 675.400 Einträge (Deutschland: 303.770, Frankreich: 342.021, Estland: 29.609) durchgeführt, um eine konsistente und signifikante Stichprobe zu erhalten.

\subsection{Variablen und Operationalisierung}\label{subsec:variablen-und-operationalisierung}
Als unabhängige Variable wird die geplante Dauer des Ausschreibungszeitraums in Tagen bemessen.
Dabei wird die Differenz zwischen dem Datum der geplanten Angebotsabgabe (endDate) und dem Datum der Veröffentlichung der Ausschreibung (publicationDate) berechnet.

Für unsere Analyse brauchen wir zwei abhängige Variablen.
Die erste beschreibt die Wettbewerbsintensität und wird direkt durch die absolute Anzahl der eingegangenen Gebote (total\_bids) pro Ausschreibung gemessen.
Die zweite abhängige Variable beschreibt die KMU-Beteiligung.
Diese wird nicht, wie bei der Wettbewerbsintensität in absoluten Zahlen gemessen sondern als prozentualer Anteil der Gesamtgebote.
Dafür rechnen wir Anzahl der KMU-Gebote (sme\_bids) durch die Gesamtzahl der Gebote (total\_bids).
Dadurch, dass wir hier den Anteil berechnen und nicht die Absolute Zahl können wir messen, ob Kleine und Mittelständige Unternehmen tatsächlich überproportional von längeren Verfahrensdauern betroffen sind oder nicht.

Für eine Robuste Analyse führen wir weitere Analysen durch mit ausgewählten Kontrollvariablen.
Zum einen analysieren wir, ob der Auftragswert (tender\_value)

\textcolor{red}{
Um den Netto-Effekt der Verfahrensdauer auf die Wettbewerbsintensität zu isolieren und Verzerrungen durch ausgelassene Variablen (Omitted Variable Bias) zu minimieren, werden in den Regressionsmodellen mehrere Kontrollvariablen berücksichtigt.
Diese Variablen wurden auf Basis der Fachliteratur zur öffentlichen Auftragsvergabe ausgewählt, da anzunehmen ist, dass sie sowohl mit der Dauer des Ausschreibungszeitraums als auch mit der Anzahl der eingegangenen Gebote korrelieren.
\begin{itemize}
\item Geschätzter Auftragswert (tender\_value): Der finanzielle Umfang einer Ausschreibung ist eine der wichtigsten Determinanten für das Bieterverhalten. Es ist davon auszugehen, dass Aufträge mit einem höheren Wert längere Angebotsfristen erhalten, um komplexere Angebote zu ermöglichen. Gleichzeitig beeinflusst der Wert die Attraktivität der Ausschreibung und damit die Anzahl der Bieter. Um eine Scheinkorrelation zwischen Dauer und Bieteranzahl zu vermeiden, wird der logarithmierte Auftragswert in das Modell aufgenommen.
\item Verfahrensart (procurement\_method): Die Art des Vergabeverfahrens hat einen direkten strukturellen Einfluss auf den Wettbewerb. Offene Verfahren (open) sind definitionsgemäß für eine unbegrenzte Anzahl von Unternehmen zugänglich und weisen oft andere Fristen auf als selektive Verfahren (selective), bei denen der Bieterkreis eingeschränkt ist. Die Kontrolle für die Verfahrensart ist daher notwendig, um diese institutionellen Rahmenbedingungen zu berücksichtigen.
\item Art der Leistung (procurement\_category): Die Wettbewerbsdynamik unterscheidet sich erheblich zwischen der Vergabe von Lieferleistungen (goods), Dienstleistungen (services) und Bauleistungen (works). Jeder dieser Sektoren hat eine eigene Marktstruktur, Anbieterdichte und typische Komplexität, was sich sowohl auf die übliche Dauer von Ausschreibungen als auch auf die zu erwartende Anzahl von Bietern auswirkt.
\item Jahr der Ausschreibung (year): Um für zeitliche Trends wie makroökonomische Zyklen, technologische Entwicklungen oder die schrittweise Wirkung von Gesetzesreformen (z.B. der EU-Vergaberichtlinien von 2014) zu kontrollieren, wird das Jahr der Veröffentlichung als kategoriale Variable in die Analyse einbezogen.
\item Land (country): Neben der Untersuchung von länderspezifischen Effekten durch Interaktionsterme dient die Ländervariable als Kontrollvariable für grundlegende, zeitinvariante Unterschiede zwischen den nationalen Beschaffungsmärkten. Faktoren wie die Wirtschaftsgröße, die Anzahl potenzieller Bieter oder die administrative Kultur können zu systematisch unterschiedlichen Wettbewerbsniveaus in Deutschland, Frankreich und Estland führen.
\item Vergabekriterien (award\_criteria): Die Kriterien für den Zuschlag beeinflussen den Aufwand für die Angebotserstellung. Eine Vergabe, die ausschließlich auf dem niedrigsten Preis basiert (priceOnly), ist für Bieter einfacher zu kalkulieren als eine, die auf dem wirtschaftlich vorteilhaftesten Angebot mit bewerteten Kriterien (ratedCriteria) beruht. Letztere erfordert komplexere Angebote, was längere Fristen rechtfertigen und gleichzeitig die Bieterzahl beeinflussen kann. Diese Variable wird daher in einer Robustheitsanalyse berücksichtigt, um die Stabilität der Ergebnisse zu prüfen.
\end{itemize}
}

\subsection{Statistische Auswertungsmethode}\label{subsec:statistische-auswertungsmethode}
% Welche statistischen Verfahren wurden genutzt? (z.B. multiple Regressionsanalyse).
\newpage
\section{Ergebnisse}\label{sec:ergebnisse}

Die empirische Analyse der Vergabedaten fördert Muster zutage, die teilweise intuitiv erscheinen, an entscheidenden Stellen jedoch gängigen Annahmen widersprechen.
Um diese Dynamiken zu verstehen, bildet eine detaillierte Betrachtung der länderspezifischen Ausgangslage das Fundament, auf dem anschließend die statistische Überprüfung der Hypothesen mittels Regressionsmodellen aufbaut.

\subsection{Deskriptive Statistiken}\label{subsec:deskriptive-statistiken}

Die deskriptive Analyse der finalen Stichprobe ($N=150.140$) offenbart deutliche strukturelle Unterschiede zwischen den drei untersuchten Ländern.
Tabelle~\ref{tab:descriptive} fasst die zentralen Kennzahlen zusammen.

Hinsichtlich der unabhängigen Variable, der Verfahrensdauer, zeigt sich, dass Deutschland mit durchschnittlich 42,69 Tagen (Median: 32) die längsten Fristen ansetzt, gefolgt von Estland (Mittelwert: 39,67; Median: 33).
Frankreich weist mit durchschnittlich 38,33 Tagen (Median: 32) die kürzesten Fristen auf, verfügt jedoch über eine geringere Standardabweichung (32,53) als Deutschland (46,79).

Auffällig ist der hohe Anteil an Ausschreibungen ohne Gebotsabgabe (Zero-Inflation).
In Frankreich erhielten 96,21\,\% der Ausschreibungen kein valides Gebot, in Estland 78,94\,\% und in Deutschland 71,96\,\%.
Dies unterstreicht die Notwendigkeit des Zero-Inflated-Modells.
Bei den erfolgreichen Ausschreibungen (Gebote $>0$) ist der Wettbewerb in Deutschland am intensivsten (Mittelwert: 1,13 Gebote), während Frankreich (0,13) und Estland (0,65) deutlich niedriger liegen.

Der Anteil der KMU an den Bietern ist in Estland mit durchschnittlich 87\,\% am höchsten, gefolgt von Deutschland (64\,\%) und Frankreich (29\,\%).

\begin{table}[htbp]
    \centering
    \caption{Deskriptive Statistiken der Stichprobe nach Ländern}
    \label{tab:descriptive}
    \begin{tabular*}{\textwidth}{l@{\extracolsep{\fill}}rrr}
        \toprule
        \textbf{Variable} & \textbf{Estland} & \textbf{Frankreich} & \textbf{Deutschland} \\
        \midrule
        \textit{Verfahrensdauer (Tage)} & & & \\
        Mittelwert (SD) & 39,67 (37,07) & 38,33 (32,53) & 42,69 (46,79) \\
        Median & 33,00 & 32,00 & 32,00 \\
        \midrule
        \textit{Wettbewerb (Anzahl Gebote)} & & & \\
        Mittelwert (SD) & 0,65 (1,62) & 0,13 (1,15) & 1,13 (3,14) \\
        Anteil ohne Gebote & 78,94\,\% & 96,21\,\% & 71,96\,\% \\
        \midrule
        \textit{KMU-Anteil (wenn Gebote $>0$)} & & & \\
        Mittelwert (SD) & 0,87 (0,28) & 0,29 (0,43) & 0,64 (0,45) \\
        \midrule
        \textit{Auftragswert (Median in \euro)} & 520.000 & 560.000 & 505.530 \\
        \bottomrule
    \end{tabular*}
    \vspace{0.1cm}
    \parbox{\textwidth}{\footnotesize{\textit{Anmerkung:} SD = Standardabweichung. Datenbasis OpenTender 2014--2022. Werte für KMU-Anteil basieren nur auf Ausschreibungen mit mindestens einem Gebot.}}
\end{table}

Zur weiteren Veranschaulichung der zentralen unabhängigen Variable zeigt Abbildung~\ref{fig:boxplot_dauer} die Verteilung der Fristen in den drei Ländern.
Die Boxplots verdeutlichen, dass die Kernverteilung (Median und Interquartilsabstand) in allen Ländern ähnlich gelagert ist, Deutschland jedoch eine tendenziell größere Streuung und mehr Ausreißer im oberen Bereich aufweist.

\begin{figure}[H]
    \centering
    \includegraphics[width=0.8\textwidth]{images/01_boxplot_dauer}
    \caption{Verteilung der Verfahrensdauer (in Tagen) nach Ländern}
    \label{fig:boxplot_dauer}
    \vspace{0.2cm}
    \footnotesize{\textit{Quelle:} Eigene Darstellung basierend auf OpenTender-Daten (Python).}
\end{figure}

\subsection{Hypothesentests}\label{subsec:hypothesentests}

\subsubsection{Einfluss der Verfahrensdauer auf die Wettbewerbsintensität}
Zur Überprüfung von Hypothese 1 (H1) und Hypothese 3 (H3) wurde ein Zero-Inflated Negative Binomial (ZINB) Modell berechnet. Die Ergebnisse sind in Tabelle \ref{tab:zinb} dargestellt.

\begin{table}[htbp]
    \centering
    \caption{Ergebnisse der ZINB-Regression auf die Anzahl der Gebote (Wettbewerbsintensität)}
    \label{tab:zinb}
    \begin{tabular*}{\textwidth}{l@{\extracolsep{\fill}}rrr}
        \toprule
        \textbf{Variable} & \textbf{Koeffizient} & \textbf{Std. Fehler} & \textbf{z-Wert} \\
        \midrule
        \multicolumn{4}{l}{\textit{Count Model (Negative Binomial)}} \\
        Intercept (Referenz: Estland) & $-3,042^{***}$ & 0,093 & -32,82 \\
        Land: Frankreich & $-1,741^{***}$ & 0,059 & -29,62 \\
        Land: Deutschland & $0,203^{***}$ & 0,059 & 3,42 \\
        \textbf{Verfahrensdauer (z-std.)} & $\mathbf{-0,026}$ & \textbf{0,063} & \textbf{-0,41} \\
        \textbf{Dauer $\times$ Frankreich} & $\mathbf{0,148^{*}}$ & \textbf{0,066} & \textbf{2,25} \\
        \textbf{Dauer $\times$ Deutschland} & $\mathbf{0,105}$ & \textbf{0,064} & \textbf{1,65} \\
        Auftragswert (log, z-std.) & $0,068^{***}$ & 0,010 & 6,75 \\
        \midrule
        \multicolumn{4}{l}{\textit{Kontrollvariablen}} \\
        Verfahrensart (Selektiv) & $-0,238^{***}$ & 0,028 & -8,54 \\
        Sektor (Dienstleistung) & $0,358^{***}$ & 0.023 & 15,55 \\
        Sektor (Bauleistung) & $0,654^{***}$ & 0,028 & 23,06 \\
        Jahres-Fixed-Effects & \multicolumn{3}{c}{Ja (signifikant)} \\
        \midrule
        \multicolumn{4}{l}{\textit{Modellgüte}} \\
        Log-Likelihood & \multicolumn{3}{c}{-95.818} \\
        Pseudo $R^2$ & \multicolumn{3}{c}{0,063} \\
        Beobachtungen ($N$) & \multicolumn{3}{c}{150.140} \\
        \bottomrule
    \end{tabular*}
    \vspace{0.1cm}
    \parbox{\textwidth}{\footnotesize{\textit{Anmerkung:} Referenzkategorie für Land ist Estland. Dauer und Wert sind z-standardisiert.\\$^{*} p < 0.05, ^{**} p < 0.01, ^{***} p < 0.001 $.}}
\end{table}

Die statistischen Ergebnisse aus Tabelle~\ref{tab:zinb} lassen sich grafisch verdeutlichen.
Abbildung~\ref{fig:interaction_wettbewerb} zeigt die vorhergesagten Werte (Predicted Margins) für die Anzahl der Gebote in Abhängigkeit von der Verfahrensdauer.
Es wird deutlich, dass die Kurve für Estland nahezu flach verläuft, was den nicht-signifikanten Effekt widerspiegelt.
Im Gegensatz dazu zeigen Deutschland und Frankreich eine sichtbare positive Steigung: Längere Verfahren gehen in diesen Ländern mit einer höheren Anzahl an Geboten einher.

\begin{figure}[H]
    \centering
    \includegraphics[width=0.75\textwidth, trim=0cm 0cm 0cm 0cm, clip]{images/02_interaction_wettbewerb_H1_H3a}
    \caption{Interaktionseffekt der Verfahrensdauer auf die vorhergesagte Anzahl an Geboten}
    \label{fig:interaction_wettbewerb}
    \vspace{0.2cm}
    \parbox{0.75\textwidth}{\footnotesize{\textit{Quelle:} Eigene Darstellung basierend auf ZINB-Modellschätzung (Python).}}
\end{figure}

Die Analyse zeigt Folgendes:
\begin{itemize}
    \item Für \textbf{Estland} (Referenz) ist der Effekt der Verfahrensdauer negativ ($-0,026$), aber statistisch nicht signifikant ($p=0,681$).
    Es besteht also kein nachweisbarer Zusammenhang zwischen Dauer und Anzahl der Gebote.
    \item Für \textbf{Frankreich} zeigt sich ein signifikant positiver Interaktionseffekt ($0,148^{*}$).
    Der Gesamteffekt ist somit positiv: Längere Verfahrensdauern führen hier tendenziell zu \textit{mehr} Geboten.
    \item Für \textbf{Deutschland} ist der Interaktionseffekt ebenfalls positiv ($0,105$), verfehlt jedoch knapp das 5\%-Signifikanzniveau ($p=0,100$).
    Der Trend deutet jedoch ebenfalls auf einen positiven Zusammenhang hin.
\end{itemize}
Hypothese 1 (länger = weniger Gebote) muss somit abgelehnt werden.
In den großen Volkswirtschaften scheint eine längere Frist eher als Chance für komplexere Angebote genutzt zu werden, statt abzuschrecken.

\subsubsection{Einfluss der Verfahrensdauer auf die KMU-Beteiligung}
Zur Überprüfung von Hypothese 2 (H2) wurde ein Fractional Logit Modell (GLM) für die Subgruppe der erfolgreichen Ausschreibungen ($N=19.760$) berechnet (siehe Tabelle~\ref{tab:glm}).

\begin{table}[htbp]
    \centering
    \caption{Ergebnisse der Fractional Logit Regression auf den KMU-Anteil}
    \label{tab:glm}
    \begin{tabular*}{\textwidth}{l@{\extracolsep{\fill}}rrr}
        \toprule
        \textbf{Variable} & \textbf{Koeffizient} & \textbf{Std. Fehler} & \textbf{z-Wert} \\
        \midrule
        Intercept (Referenz: Estland) & $-2,317^{***}$ & 0,643 & -3,61 \\
        Land: Frankreich & $-2,981^{***}$ & 0.104 & -28,67 \\
        Land: Deutschland & $-2,014^{***}$ & 0,098 & -20,57 \\
        \textbf{Verfahrensdauer (z-std.)} & $\mathbf{-0,277^{***}}$ & \textbf{0,073} & \textbf{-3,79} \\
        \textbf{Dauer $\times$ Frankreich} & $\mathbf{0,295^{***}}$ & \textbf{0,090} & \textbf{3,26} \\
        \textbf{Dauer $\times$ Deutschland} & $\mathbf{0,285^{***}}$ & \textbf{0,074} & \textbf{3,85} \\
        Auftragswert (log, z-std.) & $-0,139^{***}$ & 0,015 & -9,05 \\
        \midrule
        \multicolumn{4}{l}{\textit{Kontrollvariablen}} \\
        Verfahrensart (Selektiv) & $0,633^{***}$ & 0,044 & 14,48 \\
        Sektor (Dienstleistung) & $0,604^{***}$ & 0,043 & 14,19 \\
        Sektor (Bauleistung) & $1,125^{***}$ & 0,041 & 27,23 \\
        Jahres-Fixed-Effects & \multicolumn{3}{c}{Ja (signifikant)} \\
        \midrule
        \multicolumn{4}{l}{\textit{Modellgüte}} \\
        Log-Likelihood & \multicolumn{3}{c}{-11.222} \\
        Beobachtungen ($N$) & \multicolumn{3}{c}{19.760} \\
        \bottomrule
    \end{tabular*}
    \vspace{0.2cm}
    \parbox{\textwidth}{\footnotesize{\textit{Anmerkung:} Referenzkategorie für Land ist Estland.\\$^{*} p < 0.05, ^{**} p < 0.01, ^{***} p < 0.001 $.}}
\end{table}

Die in Tabelle~\ref{tab:glm} ausgewiesenen Interaktionseffekte werden in Abbildung~\ref{fig:interaction_kmu} visualisiert.
Die Grafik illustriert die vorhergesagte Wahrscheinlichkeit für den KMU-Anteil an den Bietern.
Hier zeigt sich der drastische Unterschied zwischen den Systemen: Während die Linien für Deutschland und Frankreich stabil verlaufen, fällt die Kurve für Estland steil ab.
Dies verdeutlicht visuell, dass Zeitverzögerungen im estnischen System eine massive Barriere speziell für kleine und mittlere Unternehmen darstellen.

\begin{figure}[H]
    \centering
    \includegraphics[width=0.75\textwidth, trim=0cm 0cm 0cm 0cm, clip]{images/03_interaction_kmu_H2_H3b}
    \caption{Interaktionseffekt der Verfahrensdauer auf den Anteil der KMU-Gebote}
    \label{fig:interaction_kmu}
    \vspace{0.2cm}
    \footnotesize{\textit{Quelle:} Eigene Darstellung basierend auf Fractional-Logit-Modellschätzung (Python).}
\end{figure}

Die Ergebnisse stützen Hypothese 2 differenziert nach Ländern:
\begin{itemize}
    \item In \textbf{Estland} zeigt sich ein hochsignifikanter negativer Effekt ($\beta = -0,277, p < 0,001$).
    Je länger das Verfahren dauert, desto geringer ist der Anteil der KMU an den Bietern.
    Hier bestätigt sich die Annahme, dass Zeitverzögerungen als Ressourcenbelastung für kleinere Unternehmen wirken.
    \item In \textbf{Deutschland} und \textbf{Frankreich} hingegen neutralisieren die signifikant positiven Interaktionseffekte ($0,285^{***}$ bzw. $0,295^{***}$) diesen negativen Basiseffekt fast vollständig.
    Der Gesamteffekt liegt nahe Null.
\end{itemize}
Zusammenfassend bestätigt sich H2 nur für das hocheffiziente System Estlands, während in Deutschland und Frankreich keine Benachteiligung der KMU durch längere Fristen nachweisbar ist.
Auch Hypothese 3 (Deutschland reagiert am stärksten negativ) muss daher abgelehnt werden; tatsächlich reagiert Estland am sensibelsten.

Diese unerwarteten Befunde – insbesondere die gegensätzlichen Effekte in Estland im Vergleich zu den großen Volkswirtschaften werfen fundamentale Fragen zur Wahrnehmung von Verfahrensdauer in unterschiedlichen Verwaltungssystemen auf, die im folgenden Kapitel diskutiert werden.
\section{Diskussion}
% Hier "bewerben" Sie Ihre Ergebnisse und zeigen, warum sie wichtig sind[cite: 2465, 2466].
% Die Diskussion folgt einer Pyramidenform (vom Spezifischen zum Allgemeinen)[cite: 2487].

\subsection{Zusammenfassung und Interpretation der Ergebnisse}
% Kurze Zusammenfassung der Hauptergebnisse in einem Absatz[cite: 2467].

\subsection{Theoretische Implikationen}
% Was bedeuten Ihre Ergebnisse für die bestehende Theorie (z.B. TKT)?[cite: 2469].
% Greifen Sie die in der Einleitung formulierten Beiträge auf[cite: 2470].

\subsection{Praktische Implikationen}
% Welche konkreten Handlungsempfehlungen ergeben sich für wen (z.B. Vergabestellen, Politik)?[cite: 2472].
% Mit Beispielen untermauern[cite: 2473].

\subsection{Limitationen und zukünftige Forschung}
% Seien Sie ehrlich über die Schwächen Ihrer Arbeit (z.B. Datenqualität, Kausalität)[cite: 2474].
% Welche Forschungsfragen ergeben sich aus Ihrer Arbeit?[cite: 2477].
\newpage
\section{Fazit}\label{sec:fazit}

Die vorliegende Bachelorarbeit untersuchte den Einfluss der geplanten Verfahrensdauer auf die Wettbewerbsintensität und die KMU-Beteiligung in der öffentlichen Auftragsvergabe in Deutschland, Frankreich und Estland.
Ziel war es, die klassische Annahme der Transaktionskostentheorie, dass Zeit primär als Kostenfaktor und Barriere wirkt, im Kontext unterschiedlicher administrativer Strukturen kritisch zu hinterfragen.

Die empirische Analyse auf Basis von Daten von Open Tender liefert ein differenziertes Ergebnis: Die zentrale Forschungsfrage nach dem Einfluss der Verfahrensdauer lässt sich nicht universell beantworten, sondern hängt maßgeblich vom institutionellen Rahmen des jeweiligen Landes ab.
Während in Deutschland und Frankreich eine längere Dauer tendenziell sogar mit einer höheren Bieterzahl korreliert, zeigt sich in Estland ein deutlicher Abschreckungseffekt auf KMU.

Daraus lässt sich das wesentliche Fazit dieser Arbeit ableiten: Die Verfahrensdauer fungiert als institutionell moderiertes Signal.
In einem hoch digitalisierten und effizienten Umfeld wie Estland wird Zeitverzögerung als administrative Anomalie und Risiko interpretiert.
In den komplexeren Systemen Deutschlands und Frankreichs hingegen wird sie eher als notwendige Ressource für die Angebotserstellung wahrgenommen.

Für die Praxis bedeutet dies, dass eine pauschale Verkürzung von Vergabefristen nicht zwingend zu mehr Wettbewerb führt.
Vielmehr müssen politische Entscheidungsträger die Signalwirkung ihrer Prozesse verstehen.
Während in Deutschland die Zentralisierung und Standardisierung zur Entlastung der Fachkräfte und zur Erhöhung der Vorhersehbarkeit im Vordergrund stehen sollte, ist in digitalen Vorreitersystemen die strikte Einhaltung von Zeitplänen essentiell, um KMU nicht systematisch zu benachteiligen.

Zukünftige Forschung sollte die hier identifizierten Signalwirkungen durch qualitative Befragungen vertiefen, um den unerforschten Aspekt der unternehmerischen Entscheidungsfindung in Abhängigkeit von Zeitparametern weiter zu entschlüsseln.
Letztlich zeigt die Arbeit, dass Bürokratieabbau nicht nur eine Frage der Geschwindigkeit, sondern vor allem der Verlässlichkeit und Transparenz administrativer Signale ist.



% --- VERZEICHNISSE (Abbildungen, Tabellen, Abkürzungen) ---

% --- Abbildungsverzeichnis ---
\newpage
\phantomsection
\addcontentsline{toc}{section}{Abbildungsverzeichnis}
\listoffigures

% --- Tabellenverzeichnis ---
\newpage
\phantomsection
\addcontentsline{toc}{section}{Tabellenverzeichnis}
\listoftables

% --- Abkürzungsverzeichnis ---
\newpage
\section*{Abkürzungsverzeichnis}
\phantomsection
\addcontentsline{toc}{section}{Abkürzungsverzeichnis}

\begin{acronym}[UGAP]
    \setlength{\itemsep}{0pt}
    \setlength{\parskip}{0pt}
    \acro{ABZ}{Auftragsberatungszentrum Bayern e. V.}
    \acro{BIP}{Bruttoinlandsprodukt}
    \acro{bpb}{Bundeszentrale für politische Bildung}
    \acro{DESI}{Digital Economy and Society Index}
    \acro{DTAD}{Deutscher Auftragsdienst}
    \acro{e-ID}{Electronic Identity (Nationale digitale ID)}
    \acro{EU}{Europäische Union}
    \acro{GLM}{Generalized Linear Model}
    \acro{GTI}{Government Transparency Institute}
    \acro{GWB}{Gesetz gegen Wettbewerbsbeschränkungen}
    \acro{HC0}{Heteroskedasticity-Consistent Standard Errors}
    \acro{JSON}{JavaScript Object Notation}
    \acro{KMU}{Kleine und mittlere Unternehmen}
    \acro{M}{Mittelwert}
    \acro{MDSD}{Most Different Systems Design}
    \acro{OECD}{Organisation for Economic Co-operation and Development}
    \acro{OLS}{Ordinary Least Squares}
    \acro{SD}{Standard Deviation (Standardabweichung)}
    \acro{UGAP}{Union des groupements d'achats publics}
    \acro{ZINB}{Zero-Inflated Negative Binomial}
    \acro{ZIP}{Zero-Inflated Poisson}
\end{acronym}

% --- LITERATURVERZEICHNIS ---
\newpage
\sloppy
\printbibliography[heading=bibintoc, title={Literaturverzeichnis}]
\fussy

% --- ANHANG ---
\newpage
\section{Anhang}
\label{sec:anhang}

% Nummerierung auf A1, A2... umstellen und Zähler zurücksetzen
\renewcommand{\thetable}{A\arabic{table}}
\setcounter{table}{0}

% --- Tabelle A1: Stichprobenreduktion (Datengrundlage zuerst) ---
\begin{table}[htbp]
    \centering
    \caption{Detaillierte Übersicht der Stichprobenreduktion nach Ländern}
    \label{tab:sample_reduction_appendix}
    \begin{tabular}{lrrr}
        \toprule
        Land & Rohdaten ($N_{initial}$) & Finale Stichprobe ($N_{final}$) & Verlust (\%) \\
        \midrule
        Deutschland & 303.770 & 55.587 & 81,7\% \\
        Frankreich & 342.021 & 91.149 & 73,3\% \\
        Estland & 29.609 & 3.404 & 88,5\% \\
        \midrule
        \textbf{GESAMT} & \textbf{675.400} & \textbf{150.140} & \textbf{77,8\%} \\
        \bottomrule
    \end{tabular}
\end{table}

% --- Tabelle A2: Korrelationsmatrix (Vorbereitende Statistik) ---
\begin{table}[htbp]
    \centering
    \caption{Mittelwerte, Standardabweichungen und Korrelationen der Kernvariablen}
    \label{tab:correlation_appendix}
    \begin{tabular}{l cc cccc}
        \toprule
        Variable & M & SD & 1 & 2 & 3 & 4 \\
        \midrule
        1. Gebotsanzahl (total\_bids) & 0,51 & 2,17 & 1,00 &  &  &  \\
        2. KMU-Anteil (sme\_share) & 0,58 & 0,46 & 0,04 & 1,00 &  &  \\
        3. Verf.-dauer (z\_duration) & -0,02 & 0,86 & 0,01 & -0,02 & 1,00 &  \\
        4. Auftragswert (z\_value) & 0,01 & 0,99 & 0,01 & -0,04 & 0,05 & 1,00 \\
        \bottomrule
    \end{tabular}
    \footnotesize Anmerkung: $N = 150.140$ (sme\_share basiert auf $n = 19.760$ für Fälle mit Gebotsanzahl $> 0$).
\end{table}

% --- Tabelle A3: Robustheitsprüfung 1 (Modellvalidierung) ---
\begin{table}[htbp]
    \centering
    \caption{Robustheitsprüfung 1: ZINB-Modell unter Einbeziehung der Zuschlagskriterien}
    \label{tab:robustness_award}
    \small
    \begin{tabular}{lcccc}
        \toprule
        Variable & Koeffizient & Std. Fehler & z-Wert & P$>|z|$ \\
        \midrule
        \textit{Count Model (Negative Binomial)} & & & & \\
        Intercept (Referenz: Estland) & -2,8796*** & 0,093 & -30,88 & 0,000 \\
        Land: Frankreich & -1,3461*** & 0,060 & -22,39 & 0,000 \\
        Land: Deutschland & 0,2006*** & 0,059 & 3,41 & 0,001 \\
        Verfahrensdauer (z-std.) & 0,0006 & 0,064 & 0,01 & 0,993 \\
        Dauer $\times$ Frankreich & 0,0997 & 0,067 & 1,49 & 0,138 \\
        Dauer $\times$ Deutschland & 0,0669 & 0,065 & 1,02 & 0,304 \\
        Auftragswert (log, z-std.) & -0,0094 & 0,010 & -0,90 & 0,370 \\
        Zuschlagskriterium (Gewichtet) & -0,4195*** & 0,028 & -14,87 & 0,000 \\ \addlinespace
        \textit{Kontrollvariablen \& Zeit-Effekte} & & & & \\
        Verfahrensart (Selektiv) & -0,1109*** & 0,028 & -3,96 & 0,000 \\
        Sektor (Dienstleistung) & 0,3857*** & 0,023 & 16,60 & 0,000 \\
        Sektor (Bauleistung) & 0,7124*** & 0,032 & 22,53 & 0,000 \\
        Jahr: 2015 & 0,3046** & 0,101 & 3,03 & 0,002 \\
        Jahr: 2016 & 1,5341*** & 0,085 & 18,02 & 0,000 \\
        Jahr: 2017 & 1,9303*** & 0,082 & 23,65 & 0,000 \\
        Jahr: 2018 & 2,3665*** & 0,080 & 29,42 & 0,000 \\
        Jahr: 2019 & 2,8567*** & 0,079 & 35,98 & 0,000 \\
        Jahr: 2020 & 2,9887*** & 0,080 & 37,23 & 0,000 \\
        Jahr: 2021 & 2,8953*** & 0,079 & 36,78 & 0,000 \\
        Jahr: 2022 & 2,5525*** & 0,079 & 32,45 & 0,000 \\
        \midrule
        Log-Likelihood & -95.818 & & & \\
        Beobachtungen (N) & 150.140 & & & \\
        \bottomrule
        \multicolumn{5}{l}{\footnotesize Anmerkung: * $p < 0,05$, ** $p < 0,01$, *** $p < 0,001$.}
    \end{tabular}
\end{table}

% --- Tabelle A4: Robustheitsprüfung 2 ---
\begin{table}[htbp]
    \centering
    \caption{Robustheitsprüfung 2: ZINB-Modell isoliert für den Dienstleistungssektor}
    \label{tab:robustness_services}
    \small
    \begin{tabular}{lcccc}
        \toprule
        Variable & Koeffizient & Std. Fehler & z-Wert & P$>|z|$ \\
        \midrule
        Intercept (Referenz: Estland) & -2,9593*** & 0,158 & -18,73 & 0,000 \\
        Land: Frankreich & -1,6864*** & 0,096 & -17,57 & 0,000 \\
        Land: Deutschland & 0,0226 & 0,098 & 0,23 & 0,818 \\
        Verfahrensdauer (z-std.) & -0,0458 & 0,112 & -0,41 & 0,683 \\
        Dauer $\times$ Frankreich & 0,1962 & 0,117 & 1,68 & 0,093 \\
        Dauer $\times$ Deutschland & 0,0685 & 0,117 & 0,59 & 0,558 \\
        Auftragswert (log, z-std.) & 0,0254 & 0,018 & 1,42 & 0,155 \\
        Verfahrensart (Selektiv) & -0,0008 & 0,037 & -0,02 & 0,984 \\
        \addlinespace
        \textit{Zeit-Fixed-Effects} & & & & \\
        Jahr: 2015 bis 2022 & \multicolumn{4}{c}{Inkludiert (Alle signifikant $p < 0,001$)} \\
        \midrule
        Beobachtungen (n) & 77.820 & & & \\
        \bottomrule
        \multicolumn{5}{l}{\footnotesize Anmerkung: Modell nur für $procurement\_category = 'services'$. * $p < 0,05$, *** $p < 0,001$.}
    \end{tabular}
\end{table}

% --- EIDESSTATTLICHE ERKLÄRUNG ---
\newpage
\newpage
\section*{Ehrenwörtliche Erklärung}

\addcontentsline{toc}{section}{\protect\numberline{}Ehrenwörtliche Erklärung}

Ich erkläre hiermit ehrenwörtlich, dass ich die vorliegende Bachelorarbeit selbstständig und ohne Benutzung anderer als der angegebenen Hilfsmittel angefertigt habe. Alle direkt oder indirekt übernommenen Quellen sind als solche im Literaturverzeichnis aufgeführt.

Mir ist bekannt, dass die Arbeit in digitaler Form auf die Verwendung unzulässiger Hilfsmittel geprüft werden kann, um festzustellen, ob die Arbeit als Ganzes oder Teile davon als Plagiat einzustufen sind. Zum Zweck des Abgleichs mit bestehenden Quellen erkläre ich mich damit einverstanden, dass die Arbeit in eine Datenbank aufgenommen wird, wo sie auch nach Abschluss der Prüfung zum Vergleich mit künftig eingereichten Arbeiten verbleibt. Weitere Nutzungs- und Vervielfältigungsrechte werden hierdurch jedoch nicht eingeräumt.

Diese Arbeit hat in gleicher oder ähnlicher Form noch keiner anderen Prüfungsbehörde vorgelegen und wurde bisher nicht veröffentlicht.

\vspace{2cm}

\noindent
\begin{tabular}{@{}l p{2cm} l@{}}
München, den \today & & \rule{6cm}{0.5pt} \\
Ort, Datum & & Sofia Schepers
\end{tabular}
% ...

\end{document}